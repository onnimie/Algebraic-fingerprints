\documentclass[12pt,a4paper,english,oneside]{article}

% Valitse 'input encoding':
%\usepackage[latin1]{inputenc} % merkistökoodaus, jos ISO-LATIN-1:tä.
\usepackage[utf8]{inputenc}   % merkistökoodaus, jos käytetään UTF8:a
% Valitse 'output/font encoding':
%\usepackage[T1]{fontenc}      % korjaa ääkkösten tavutusta, bittikarttana
\usepackage{ae,aecompl}       % ed. lis. vektorigrafiikkana bittikartan sijasta
% Kieli- ja tavutuspaketit:
%\usepackage[finnish]{babel}
% Muita paketteja:
% \usepackage{amsmath}   % matematiikkaa
\usepackage{url}       % \url{...}

% Kappaleiden erottaminen ja sisennys
\parskip 1ex
\parindent 0pt
\evensidemargin 0mm
\oddsidemargin 0mm
\textwidth 159.2mm
\topmargin 0mm
\headheight 0mm
\headsep 0mm
\textheight 246.2mm

\pagestyle{plain}

% ---------------------------------------------------------------------

\begin{document}

% Otsikkotiedot: muokkaa tähän omat tietosi

\title{CS. BsC. Thesis Research Plan:\\[5mm]
Algebraic Fingerprinting}

\author{Onni Miettinen\\
Aalto-university, SCI\\
\url{onni.miettinen@aalto.fi}}

\date{\today}

\maketitle

% ---------------------------------------------------------------------

\vspace{10mm}

% MUOKKAA TÄHÄN. Jos tarvitset tähän viitteitä, käytä
% tässä dokumentissa numeroviitejärjestelmää komennolla \cite{kahva}.
%
% Paljon kandidaatintöitä ohjanneen Vesa Hirvisalon tarjoama 
% sabluuna. Kursivoidut osat \emph{...} ovat kurssin henkilökunnan
% lisäämiä. 

\textbf{Name of the thesis:} Algebraic Fingerprinting

\textbf{Author:} Onni Miettinen

\textbf{Thesis advisor:} Augusto Modanese


\section{Summary of research}

This thesis is a literature review on the use of algebraic fingerprinting in faster algorithms for parameterized problems.

\section{Research goals and perspectives}

The main goal is to understand how faster algorithms can be designed with the utilization of algebraic fingerprinting.
The thesis could cover these perspectives: 
(a) what is algebraic fingerprinting,
(b) implementing algebraic fingerprinting techniques,
(c) problem specific solutions to linear monomial detection

\section{Research materials}

The research materials consist of published papers and articles on the topic.
There seems to be a lot of material related to the subject.

The thesis research consists of mainly reading through the papers. Without thorough investigation,
I would estime the research costs as at least a day or two per paper. However, this is a hard estime
since I will not go through the material sequentially when researching.

\section{Methods of research}

To acquire material, I plan to use Google Scholar, Web of Sciences, Scopus and other academic databases.
The idea is to find a paper, locate key terms, ideas and problems, and search for further material using the acquired knowledge.
Collecting a lot of material should allow for specifying the focus of the research and thesis.

With the research converging on an area of interest, the relevant material is reviewed and further investigated.

After this, the relevant information and sources are organized and further studied.
To help with studying, I will use methods such as mindmaps and own notes, explaining concepts to roommate with own words,
relating every information piece with each other.

The organization and relations of information should allow for structuring a report of the research, which will be the thesis.

\section{Obstacles}

The method of converging on an area of interest is problematic when the behaviour of the interest function is unknown on the domain.
This is the case here, since the problem domain of algebraic fingerprinting is new to me. This may bring multiple problems:
(1) the function never converges on a specific area,
(2) the convergence is really slow and costly in relation reading stamina and work,
(3) it is hard to evaluate when this method has converged enough, i.e., the area of interest is small enough for a satisfactory scope.

Another obstacle for research is of course my own schedule, which will be really busy and, even worse, very varying.
This means that the weekly workload will not be consistent, which can bring problems with course deadlines.

\section{Resources}

The research itself is done alone by me. Augusto Modanese is my thesis advisor and can help with understanding the topic itself as well
as writing the thesis (methods and language of reporting). With the university, I have access to large databases of papers on the topic.

\section{Schedule draft}

This is a draft of a schedule, since it is impossible to create one that can be realistically followed with current information.

\begin{tabular}{|p{30mm}|p{120mm}|}
\hline
week no.   & done by the end of the week \\ \hline
4   & research plan \\ \hline
5   & established some scope for thesis \\ \hline
6   & Version 1 (2-3 pages: some introduction, table of contents or composition/structure. At least 1 reference) \\ \hline
7   & further research with review of scope/topic \\ \hline
8   & logical structure for discussion in thesis and some content (headlines and their order for example) \\ \hline
9   & Version 2 (~10 pages: headline structure (with subheadlines), introduction complete, some important content in) \\ \hline
10   & research \\ \hline
11   & research \\ \hline
12   & Version 3 (aim for final) \\ \hline
\end{tabular}


\section{Table of contents draft}

(1) Introduction,
(2) Algebraization of problems,
(3) Methods for detecting linear monomials,
(4) Summary.
%

\emph{This structure and the headlines will very likely change with further research into the subject.}

% ---------------------------------------------------------------------
%
% ÄLÄ MUUTA MITÄÄN TÄÄLTÄ LOPUSTA

% Tässä on käytetty siis numeroviittausjärjestelmää. 
% Toinen hyvin yleinen malli on nimi-vuosi-viittaus.

% \bibliographystyle{plainnat}
%\bibliographystyle{finplain}  % suomi
\bibliographystyle{plain}    % englanti
% Lisää mm. http://amath.colorado.edu/documentation/LaTeX/reference/faq/bibstyles.pdf

% Muutetaan otsikko "Kirjallisuutta" -> "Lähteet"
\renewcommand{\refname}{Lähteet}  % article-tyyppisen

% Määritä bib-tiedoston nimi tähän (eli lahteet.bib ilman bib)
\bibliography{lahteet}

% ---------------------------------------------------------------------

\end{document}
