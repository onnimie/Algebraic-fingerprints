% Tiivistelmät tehdään viimeiseksi. 
%
% Tiivistelmä kirjoitetaan käytetyllä kielellä (JOKO suomi TAI ruotsi)
% ja HALUTESSASI myös samansisältöisenä englanniksi.
%
% Avainsanojen lista pitää merkitä main.tex-tiedoston kohtaan \KEYWORDS.

\begin{fiabstract}
  Tässä työssä tutkitaan, kuinka parametrisoituja kombinatorisia ongelmia voidaan 
  ratkaista %löytämällä $k$-asteinen multilineaarinen monomi monimuuttujaisesta polynomista 
  algebrallisin keinoin. Kandidaatintyö keskittyy tarkemmin 
  Williamsin ja Koutisin algebrallisia sormenjälkiä hyödyntävään 
  tekniikkaan, jolla voidaan havaita multilineaarinen monomi tehokkaasti. 
  Työ tutkii tekniikan taustalla olevaa algebraa ja sen mahdollisia rajoja,  
  kun sitä tarkastellaan yleisenä kehyksenä parametrisoitujen ongelmien ratkaisemiseksi. 
  Viimeiseksi pohditaan keinoja, joilla tätä kehystä voitaisiin kehittää tai jopa 
  ohittaa sen antama raja algoritmien tehokkuudelle. 

  Kombinatoriset ongelmat voidaan redusoida algebralliseen muotoon, jossa 
  pyritään havaitsemaan multilineaarinen monomi monimuuttujaisesta polynomista. 
  Algebrallisten sormenjälkien avulla multilineaarinen monomi 
  voidaan havaita ajassa \bigO{2^k \cdot \poly(n)}, 
  jossa $n$ on $k$-asteisen polynomin muuttujien lukumäärä. Tämä johtaa yleiseen kehykseen, 
  jolla parametrisoituja kombinatorisia ongelmia voidaan ratkaista. 
  Tällä hetkellä algebrallisten sormenjälkien 
  tekniikka pohjaakin nopeimpia algoritmeja monille parametrisoiduille 
  kombinatorisille ongelmille.

  \bigOmega{2^k \cdot \poly(n)} on kuitenkin 
  kehyksen alaraja: algebrallisten sormenjälkien 
  avulla ei voida yleisesti havaita multilineaarisia monomeja nopeammin. Toisaalta 
  kehys kohdistuu hyvin yleismuotoiseen multilineaaristen monomien havaitsemisongelmaan; 
  jos hyödynnetään ongelmakohtaisia kombinatorisia ominaisuuksia, 
  voidaan löytää algoritmi, joka ratkaisee ongelman nopeammin. 
  Muun muassa Björklund kehitti 
  algebrallisten sormenjälkien antamien ideoiden avulla algoritmin, joka 
  löytää Hamiltonisen polun ajassa \bigO{1.657^n \cdot \poly(n)}. 
  Näyttääkin siltä, että algebrallisten sormenjälkien syvä ymmärrys sekä ongelmakohtainen 
  hyödyntäminen ovat avaimia nopeammille algoritmeille ja kehitykselle yleisestä algebrallisesta 
  kehyksestä.
%
%Tiivistelmätekstiä tähän (\languagename). Huomaa, että tiivistelmä tehdään %vasta kun koko työ on muuten kirjoitettu.

\end{fiabstract}
