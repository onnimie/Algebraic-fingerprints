% Tiivistelmät tehdään viimeiseksi. 
%
% Tiivistelmä kirjoitetaan käytetyllä kielellä (JOKO suomi TAI ruotsi)
% ja HALUTESSASI myös samansisältöisenä englanniksi.
%
% Avainsanojen lista pitää merkitä main.tex-tiedoston kohtaan \KEYWORDS.

\begin{fiabstract}
  Tässä työssä tutkitaan, kuinka parametrisoituja kombinatorisia ongelmia voidaan 
  ratkaista löytämällä $k$-asteinen multilineaarinen monomi monimuuttujaisesta 
  polynomista algebrallisin keinoin. Kandidaatintyö keskittyy tarkemmin 
  Williamsin ja Koutisin algebrallisia sormenjälkiä hyödyntävään 
  tekniikkaan, jolla voidaan havaita multilineaarinen monomi tehokkaasti. 
  Lisäksi tutkitaan tämän tekniikan suomaa yleistä kehystä parametrisoitujen 
  ongelmien ratkaisemiseksi ja sen mahdollisia rajoja. 
  Viimeiseksi pohditaan keinoja, joilla kehystä voitaisiin kehittää tai jopa 
  ohittaa kehyksen 
  antama aliraja algoritmien tehokkuudelle. 

  Williamsin ja Koutisin algebrallisten sormenjälkien avulla voidaan 
  havaita $k$-asteen multilineaarinen monomi ajassa \bigO{2^k * poly(n)}, 
  jossa $n$ on polynomin muuttujien lukumäärä. Tämä johtaa yleiseen 
  parametrisoitujen ongelmien kehykseen, jolla niitä voidaan ratkaista 
  tehokkaasti. 
  Etenkin color-coding—metodiin 
  nojaavia algoritmeja voidaan nopeuttaa tekijällä $e^k$ hyödyntämällä 
  algebrallisia sormenjälkiä. Tällä hetkellä algebrallisten sormenjälkien 
  tekniikka pohjaa nopeimpia algoritmeja monille parametrisoiduille 
  kombinatorisille ongelmille.

  \bigOmega{2^k * poly(n)} on kuitenkin 
  kehyksen alaraja: Williamsin ja Koutisin algebrallisten sormenjälkien 
  avulla ei voida havaita multilineaarisia monomeja nopeammin. Toisaalta 
  heidän kehys perustuu yleiseen multilineaaristen monomien havaitsemisongelmaan; 
  jos hyödynnetään ongelmakohtaisia kombinatorisia ominaisuuksia ongelman 
  redusoinnissa multilineaaristen monomien havaitsemiseksi, 
  voidaan löytää algoritmi, joka suoriutuu nopeammin kyseisellä ongelmalla. Björklund kykeni 
  kehittämään algebrallisten sormenjälkien avulla algoritmin, joka 
  löytää Hamiltonisen polun ajassa \bigO{1.657^n * poly(n)}.
%
%Tiivistelmätekstiä tähän (\languagename). Huomaa, että tiivistelmä tehdään %vasta kun koko työ on muuten kirjoitettu.

\end{fiabstract}
