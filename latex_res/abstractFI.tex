% Tiivistelmät tehdään viimeiseksi. 
%
% Tiivistelmä kirjoitetaan käytetyllä kielellä (JOKO suomi TAI ruotsi)
% ja HALUTESSASI myös samansisältöisenä englanniksi.
%
% Avainsanojen lista pitää merkitä main.tex-tiedoston kohtaan \KEYWORDS.

\begin{fiabstract}
  Tässä työssä tutkitaan, kuinka kombinatorisia ongelmia voidaan 
  ratkaista %löytämällä $k$-asteinen multilineaarinen monomi monimuuttujaisesta polynomista 
  algebrallisin keinoin. Kandidaatintyö keskittyy tarkemmin 
  Koutisin (ICALP 2008) ja Williamsin (ICALP 2009) algebrallisia sormenjälkiä 
  hyödyntävään tekniikkaan. 
  Työ tutkii tekniikan taustalla olevaa algebraa ja sen mahdollisia rajoja,  
  kun sitä tarkastellaan yleisenä kehyksenä kombinatoristen ongelmien ratkaisemiseksi 
  parametrisoidussa kompleksisuudessa. 
  Viimeiseksi työssä pohditaan keinoja, joilla tätä kehystä voidaan kehittää tai joilla 
  voidaan ohittaa sen antama raja algoritmien tehokkuudelle. 

  Kombinatoriset ongelmat voidaan redusoida algebralliseen muotoon, jossa 
  pyritään havaitsemaan multilineaarinen monomi monimuuttujaisesta polynomista. 
  Algebrallisten sormenjälkien avulla multilineaarinen monomi 
  voidaan havaita ajassa \bigO{2^k \cdot \poly(n)}, 
  jossa $n$ on $k$-asteisen polynomin muuttujien lukumäärä. 
  Tämä toimii yleisen kehyksenä, 
  jolla kombinatorisia ongelmia voidaan ratkaista parametrisoidussa kompleksisuudessa. 
  Tällä hetkellä algebrallisten sormenjälkien 
  tekniikka pohjaa nopeimpia algoritmeja monille 
  kombinatorisille ongelmille.

  \bigOmega{2^k \cdot \poly(n)} on kuitenkin 
  kehyksen alaraja: tekniikalla ei voida yleisesti havaita multilineaarisia monomeja nopeammin. Toisaalta 
  kehys kohdistuu hyvin yleismuotoiseen multilineaaristen monomien havaitsemisongelmaan; 
  jos hyödynnetään ongelmakohtaisia kombinatorisia ominaisuuksia, 
  voidaan löytää algoritmi, joka ratkaisee ongelman nopeammin. 
  Muun muassa Björklund (FOCS 2010) kehitti 
  algebrallisten sormenjälkien antamien ideoiden avulla algoritmin, joka 
  löytää Hamiltonisen polun ajassa \bigO{1.657^n \cdot \poly(n)}. 
  Näyttääkin siltä, että algebrallisten sormenjälkien syvä ymmärrys sekä ongelmakohtainen 
  hyödyntäminen ovat avaimia nopeammille algoritmeille.
%
%Tiivistelmätekstiä tähän (\languagename). Huomaa, että tiivistelmä tehdään %vasta kun koko työ on muuten kirjoitettu.

\end{fiabstract}
