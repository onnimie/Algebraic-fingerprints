Although color coding is outperformed by the 
purely algebraic technique of fingerprinting, color coding is given as valuable background information, 
since some ideas carry on to the algebraic methods. Moreover, 
it appears that most parameterized problems that are solved with color coding can be reduced to 
multilinear monomial detection, and thus have faster algorithms
%\cite{KouWil15}, \cite{Fomin17}.
\amnote*{\texttt{\textbackslash cite} can take multiple parameters, see source}{
\cite{KouWil15,Fomin17}.
}

\amnote*{Use \texttt{\textbackslash label} and \texttt{\textbackslash cref}}{
In a later subsection, 
the thesis covers color coding with matrices as a purely algebraic method, which leads into 
the technique of algebraic fingerprinting. Then, a general framework for algebraic fingerprinting is discussed. 
In the last subsection under this section, 
the thesis discusses some limits and cons in general multilinear monomial
detection.
}

%However, a faster algorithm for some instance of multilinear monomial detection does not directly imply faster 
%algorithms for other combinatorial problems, since the algebraization into multilinear monomial detection is 
%problem specific. On the other hand, many multilinear monomial detection problems employ similar techniques [TODO: quick examples (just refs?)]. 
%As a result, finding new techniques and ideas for some specific multilinear monomial detection is valuable. 
%This thesis discusses clever utilizations of algebraic fingerprints on specific problems in a later section. 

\subsection{Non-algebraic methods}

Before discussing color coding, the thesis briefly overviews dynamic programming in the context of 
multilinear monomial detection, since dynamic programming is used in color coding. Furthermore, 
\amnote*[inline,nomargin]{we discuss}{the thesis introduces}
an algebraically interesting idea of setting squared variables to zero.

\subsubsection{Dynamic programming for smart expansion of the polynomial}

Multilinear monomial detection can be solved with dynamic programming. 
Dynamic programming, used for e.g. Hamiltonicity in \cite{HelKar62}, is a method where the problem 
is recursively broken down into smaller subproblems. \cite{Fomin17} gives a dynamic programming algorithm 
for solving an instance of $k$-multilinear monomial detection in the problem context of 
\amnote*{??}{representative families for product families}.

In multilinear monomial detection, only the multilinear terms are important. This implies that any non-linear term can be discarded 
as soon as they are formed in the arithmetic circuit, since the arithmetic circuit will never decrease the degree of a monomial. 
Therefore, a smart algorithm can be designed for the expansion that applies an additional rule: 
any squared variable can be instantly discarded.

With dynamic programming, this can be implemented by redesigning the arithmetic gates in the arithmetic circuit. 
In \amnote*[inline,nomargin]{practice}{practise},
\amnote*{I am not quite sure what is going on here}{%
the algorithm would employ
special addition and multiplication gates that discard squared variables. 
Thus, the evaluation of the (sum of monomials form) polynomial is reduced to
just evaluating the multilinear terms. 
For a multivariate polynomial with $n$ variables, there are $2^n$ multilinear terms, which implies 
that the expansion can be ran in \bigOstar{2^n} time \cite{KouWil15}.
}

The rule of discarding squared variables can be written in algebraic form: 
a squared variable $x^2 \in X$ is set to the additive identity (zero) of the field $\textbf F$, 
where $X \subset F$. 
With this additional rule for multilinear monomial detection, the following is deduced: 
if there are no solutions to the original problem, 
the polynomial will identically evaluate to zero.
\amnote[inline,nomargin]{So algebraically we are computing over
$\mathbf{F}[x]/(x^2)$?}

Note, however, that the polynomial evaluating to zero does not
\amnote[inline,nomargin]{necessarily}
imply that no
solutions exist. 
Indeed, we will come across a problem where multilinear monomials cancel each other, 
and thus evaluate to zero (see Section X.X).

\subsubsection{Non-deterministic color coding for faster evaluation}

\amnote{I am not sure if this is not a too big of a detour}

TODO: go over random assignment with colors, 
which results a polynomial of smaller domain (less variables), thus 
making the evaluation faster

Color coding was introduced in \cite{Alon95} as a randomized method for subgraph problems. Among others, 
a non-deterministic algorithm for the $k$-path problem is given.
First, the thesis explains the idea behind the algorithm in \cite{Alon95}. 
After this, the idea is translated into multilinear monomial detection for easier relevance.

\begin{problem}
  \problemtitle{\textsc{(directed or undirected)} $k$-\textsc{path (decision and search)}}
  \probleminput{A (directed or undirected) graph $G=(V,E)$.}
  \problemquestion{Does $G$ contain a simple path on $k$ vertices? If so, give such a path.}
\end{problem}

TODO: go quickly over the algorithm, translate it in terms of multilinear monomial detection, 
give focus on the random assignment in order to reduce domain size (adds non-determinism, though)

TODO: end with hints toward algebraic assignment (for example, a non-zero matrix squared can equal zero)
