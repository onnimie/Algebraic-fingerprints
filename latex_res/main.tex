% ---------------------------------------------------------------------
% -------------- PREAMBLE ---------------------------------------------
% ---------------------------------------------------------------------
\documentclass[12pt,a4paper,finnish,oneside]{article}
%\documentclass[12pt,a4paper,finnish,twoside]{article}
%\documentclass[12pt,a4paper,finnish,oneside,draft]{article} % luonnos, nopeampi

% Valitse 'input encoding':
%\usepackage[latin1]{inputenc} % merkistökoodaus, jos ISO-LATIN-1:tä.
\usepackage[utf8]{inputenc}   % merkistökoodaus, jos käytetään UTF8:a
% Valitse 'output/font encoding':
%\usepackage[T1]{fontenc}      % korjaa ääkkösten tavutusta, bittikarttana
\usepackage{ae,aecompl}       % ed. lis. vektorigrafiikkana bittikartan sijasta
% Kieli- ja tavutuspaketit:
\usepackage[english,swedish,finnish]{babel}
% Kurssin omat asetukset aaltosci_t.sty:
%\usepackage{aaltosci_t}
% Jos kirjoitat muulla kuin suomen kielellä valitse:
%\usepackage[finnish]{aaltosci_t}           
%\usepackage[swedish]{aaltosci_t}           
\usepackage[english]{aaltosci_t}           
% Muita paketteja:
\usepackage{alltt}
\usepackage{amsmath}   % matematiikkaa
\usepackage{calc}      % käytetään laskurien (counter) yhteydessä (tiedot.tex)
\usepackage{eurosym}   % eurosymboli: \euro{}
\usepackage{url}       % \url{...}
\usepackage{listings}  % koodilistausten lisääminen
\usepackage{algorithm} % algoritmien lisääminen kelluvina
\usepackage{algorithmic} % algoritmilistaus
\usepackage{hyphenat}  % tavutuksen viilaamiseen liittyvä (hyphenpenalty,...)
\usepackage{supertabular,array}  % useampisivuinen taulukko

\usepackage{microtype}  % better line breaks
\usepackage{csquotes}   % \enquote{}
\usepackage{cleveref}   % \cref{}
\usepackage{complexity} % \poly
\usepackage{amsthm}     % for lemma

%\usepackage{amsfonts}
\usepackage{tabularx,environ,amssymb}
\usepackage[
backend=biber,
style=alphabetic,
sorting=nyt,
maxbibnames=99,  % prevents "et al." in bibliography
sortcites  % sorts citations
]{biblatex}
\addbibresource{sources.bib}
\usepackage[rightcaption]{sidecap}
\usepackage{graphicx} % for images

% For comments (remove from final version)
\usepackage[draft]{fixme}
\fxusetheme{color}
\setlength{\marginparwidth}{2cm}
\FXRegisterAuthor{am}{anam}{AM}
\FXRegisterAuthor{om}{anom}{OM} % if you want to add comments too

% Koko dokumentin kattavia asetuksia:
\graphicspath{ {./images} } % give path to images for use in figures

% Tavutettavia sanoja:
%\hyphenation{vää-rin me-ne-vi-en eri-kois-ten sa-no-jen tavu-raja-ehdo-tuk-set}
% Huomaa, että ylläoleva etsii tarkalleen kyseisiä merkkijonoja, eikä
% ymmärrä taivutuksia. Paikallisesti tekstin seassa voi myös ta\-vut\-taa.

% Rangaistaan tavutusta (ei toimi?! Onko hyphenat-paketti asennettu?)
\hyphenpenalty=10000   % rangaistaan tavutuksesta, 10000=ääretön
\tolerance=1000        % siedetään välejä riveillä
% titlesec-paketti auttaa, jos tämän mukana menee sekaisin

% Tekstiviitteiden ulkoasu.
% Pakettiin natbib.sty/aaltosci.bst liittyen katso esim. 
% http://merkel.zoneo.net/Latex/natbib.php
% jossa selitykset citep, citet, bibpunct, jne.
% Valitse alla olevista tai muokkaa:
%\bibpunct{(}{)}{;}{a}{,}{,}    % a = tekijä-vuosi (author-year)
%\bibpunct{[}{]}{;}{n}{,}{,}    % n = numero [1],[2] (numerical style)

% Rivivälin muuttaminen:
\linespread{1.24}\selectfont               % riviväli 1.5
%\linespread{1.24}\selectfont               % riviväli 1, kun kommentoit pois

% ---------------------------------------------------------------------
% -------------- DOCUMENT ---------------------------------------------
% ---------------------------------------------------------------------

\begin{document}

% -------------- Tähän dokumenttiin liittyviä valintoja  --------------

%\raggedright         % Tasattu vain vasemmalta, ei tavutusta
% ----------------- joitakin makroja ----------------------------------
%
% \newcommand{\sinunKomentosi}[argumenttienMäärä]{komennot%
% voiJakaaRiveille%
% jaArgumenttienViittaus#1,#2,#argumenttienMäärä}

% Joskus voi olla tarpeen kommentoida jotakin. Ei suositella. 
% Äläkä unohda lopulliseen! 
\newcommand{\Kommentti}[1]{\fbox{\textbf{OMA KOMMENTTI:} #1}}
% Käyttö: Kilometri on 1024 metriä. \Kommentti{varmista tämä vielä}.
% Eli newcommand:n komentosanan jälkeen hakasaluissa argumenttien lkm,
%  ja argumentteihin viitataa #1, #2, ...

%  Comment out this \DRAFT macro if this version no longer is one!  XXX
%\newcommand{\DRAFT}{\begin{center} {\it DRAFT! \hfill --- \hfill DRAFT!
%\hfill --- \hfill DRAFT! \hfill --- \hfill DRAFT!}\end{center}}

%  Use this \DRAFT macro in the final version - or comment out the 
%  draft-command
% \newcommand{\DRAFT}{~}

% %%%%%%%% MATEMATIIKKA %%%%%%%%%%%%%%%%%

% Määrätty integraali
\newcommand{\myInt}[4]{%
\int_{#1}^{#2} #3 \, \textrm{d}{#4}}

% http://kapsi.fi/jks/satfaq/
%\newcommand{\vii}{\mathop{\Big/}}
%\newcommand{\viiva}[2]{\vii\limits_{\!\!\!\!{#1}}^{\>\,{#2}}}
%%\[ \intop_0^{10} \frac{x}{x^2+1} \,\mathrm{d}x
%%= \viiva{0}{10} \frac{1}{2}\ln(x^2+1) \]

% matht.sty, Simo K. Kivelä, 01.01.2002, 07.04.2004, 19.11.2004, 21.02.2005
% Kokoelma matemaattisten lausekkeiden kirjoittamista helpottavia
% määrittelyjä.

% 07.04.2004 Muutama lisäys ja muutos tehty: \ii, \ee, \dd, \der,
% \norm, \abs, \tr.
%
% 19.11.2004 Korjattu määrittelyjä: \re, \im, \norm;
% lisätty \trp (transponointi), \hrm (hermitointi), \itgr (rakenteellinen
% integraali), ympäristö Cmatrix (hakasulkumatriisi);
% vanha transponointi \tr on mukana edelleen, mutta ei suositella.

% Pakotettu rivinvaihto, joka voidaan tarvittaessa määritellä
% uudelleen: 

\newcommand{\nl}{\newline}

% Logiikan symboleja (<=> ja =>) hieman muunnettuina:

%\newcommand{\ifftmp}{\;\Leftrightarrow\;}
%\newcommand{\impltmp}{\DOTSB\;\Rightarrow\;}

% 'siten, että' -lyhenne ja hattupääyhtäläisyysmerkki vastaavuuden
% osoittamiseen: 

%\newcommand{\se}{\quad \text{siten, että} \quad}
%\newcommand{\vs}{\ {\widehat =}\ }

% Lukujoukkosymbolit:

\newcommand{\N}{\ensuremath{\mathbb N}}
\newcommand{\Z}{\ensuremath{\mathbb Z}}
\newcommand{\Q}{\ensuremath{\mathbb Q}}
\newcommand{\R}{\ensuremath{\mathbb R}}
\newcommand{\C}{\ensuremath{\mathbb C}}

\newcommand{\bigO}[1]{$\mathcal{O}(#1)$}

% Reaali- ja imaginaariosa, imaginaariyksikkö:

%\newcommand{\re}{\operatorname{Re}}
%\newcommand{\im}{\operatorname{Im}}
%\newcommand{\ii}{\mathrm{i}}

% Differentiaalin d, Neperin luku:

%\newcommand{\dd}{\mathrm{d}}
%\newcommand{\ee}{\mathrm{e}}

% Vektorimerkintä, joka voidaan tarvittaessa määritellä uudelleen
% (tämä tekee vektorit lihavoituina):

%\newcommand{\V}[1]{{\mathbf #1}}

% Kulmasymboli:

%\renewcommand{\angle}{\sphericalangle}

% Vektorimerkintä, jossa päälle pannaan iso nuoli;
% esimerkiksi \overrightarrow{AB} (tämmöisiä olemassaolevien
% symbolien uudelleenmäärittelyjä ei kyllä pitäisi tehdä):

%\renewcommand{\vec}[1]{\overrightarrow{#1}}

% Vektoreiden vastakkaissuuntaisuus:

%\newcommand{\updownarrows}{\uparrow\negthinspace\downarrow}

% Itseisarvot ja normi:

\newcommand{\abs}[1]{{\left\vert#1\right\vert}}
%\newcommand{\norm}[1]{{\left\Vert #1 \right\Vert}}

% Transponointi ja hermitointi:

%\newcommand{\trp}[1]{{#1}\sp{\operatorname{T}}}
%\newcommand{\hrm}[1]{{#1}\sp{\operatorname{H}}}

% Vanha transponointi; jäljellä yhteensopivuussyistä, ei syytä käyttää.
%\newcommand{\tr}{{}^{\text T}}

% Arcus- ja area-funktiot, jossa päähaara osoitetaan nimen päälle
% vedetyllä vaakasuoralla viivalla (alkaa olla vanhentunutta,
% voitaisiin luopua):

%\newcommand{\arccot}{\operatorname{arccot}}
%\newcommand{\asin}{\operatorname{\overline{arc}sin}}
%\newcommand{\acos}{\operatorname{\overline{arc}cos}}
%\newcommand{\atan}{\operatorname{\overline{arc}tan}}
%\newcommand{\acot}{\operatorname{\overline{arc}cot}}

%\newcommand{\arsinh}{\operatorname{arsinh}}
%\newcommand{\arcosh}{\operatorname{arcosh}}
%\newcommand{\artanh}{\operatorname{artanh}}
%\newcommand{\arcoth}{\operatorname{arcoth}}
%\newcommand{\acosh}{\operatorname{\overline{ar}cosh}}

% Signum, syt, pyj:

%\newcommand{\sg}{\operatorname{sgn}}
%\renewcommand{\gcd}{\operatorname{syt}}
%\newcommand{\lcm}{\operatorname{pyj}}

% Lyhennemerkintöjä: derivaatta, osittaisderivaatta, gradientti,
% derivaattaoperaattori, vektorin komponentti, integraalin ylä- ja
% alasumma, Suomessa (ja Saksassa?) käytetty integraalin sijoitus-
% merkintä, integraali (rakenteellinen määrittely):

%\newcommand{\der}[2]{\frac{\dd #1}{\dd #2}}
%\newcommand{\osder}[2]{\frac{\partial #1}{\partial #2}}
%\newcommand{\grad}{\operatorname{grad}}
%\newcommand{\Df}{\operatorname{D}} 
%\newcommand{\comp}{\operatorname{comp}}
%\newcommand{\ys}[1]{\overline S_{#1}}
%\newcommand{\as}[1]{\underline S_{#1}}
%\newcommand{\sijoitus}[2]{\biggl/_{\null\hskip-6pt #1}^{\null\hskip2pt #2}} 
%\newcommand{\itgr}[4]{\int_{#1}^{#2}#3\,\dd #4}

% Matriiseja, joille voidaan antaa alkioiden sijoittamista sarakkeen
% vasempaan tai oikeaan reunaan tai keskelle osoittava lisäparametri
% (l, r tai c); ympärillä kaarisulut, hakasulut, pystyviivat (determinantti)
% tai ei mitään;
% esimerkiksi \begin{cmatrix}{ll}1 & -1 \\ -1 & 1 \end{cmatrix}:

%\newenvironment{cmatrix}[1]{\left(\begin{array}{#1}}{\end{array}\right)}
%\newenvironment{Cmatrix}[1]{\left[\begin{array}{#1}}{\end{array}\right]}
%\newenvironment{dmatrix}[1]{\left|\begin{array}{#1}}{\end{array}\right|}
%\newenvironment{ematrix}[1]{\begin{array}{#1}}{\end{array}}

% Kaunokirjoitussymboli:

%\newcommand{\Cal}{\mathcal}

% Isokokoinen summa:

%\newcommand{\dsum}[2]{{\displaystyle \sum_{#1}^{#2}}}

% Tuplaintegraali umpinaisen pinnan yli; korvataan jos parempi löytyy:
%\newcommand\oiint{\begingroup
% \displaystyle \unitlength 1pt
% \int\mkern-7.2mu
% \begin{picture}(0,3)
%   \put(0,3){\oval(10,8)}
% \end{picture}
% \mkern-7mu\int\endgroup}
       % Haetaan joitakin makroja

% Kieli:
% Kielesi, jolla kandidaatintyön kirjoitat: finnish, swedish, english.
% Tästä tulee mm. tietyt otsikkonimet ja kuva- ja taulukkoteksteihin 
% (Kuva, Figur, Figure), (Taulukko, Tabell, Table) sekä oikea tavutus.
%\selectlanguage{finnish}
%\selectlanguage{swedish}
\selectlanguage{english}

% Sivunumeroinnin kanssa pieniä ristiriitaisuuksia.
% Toimitaan pääosin lähteen "Kirjoitusopas" luvun 5.2.2 mukaisesti.
% Sivut numeroidaan juoksevasti arabialaisin siten että 
% ensimmäiseltä nimiölehdeltä puuttuu numerointi.
\pagestyle{plain}
\pagenumbering{arabic}
% Muita tapoja: kandiohjeet: ei numerointia lainkaan ennen tekstiosaa
%\pagestyle{empty}
% Muita tapoja: kandiohjeet: roomalainen numerointi alussa ennen tekstiosaa
%\pagestyle{plain}
%\pagenumbering{roman}        % i,ii,iii, samalla alustaa laskurin ykköseksi

% ---------------------------------------------------------------------
% -------------- Luettelosivut alkavat --------------------------------
% ---------------------------------------------------------------------

% -------------- Nimiölehti ja sen tiedot -----------------------------
%
% Nimiölehti ja tiivistelmä kirjoitetaan seminaarin mukaan joko
% suomeksi tai ruotsiksi (ellei erityisesti kielenä ole englanti). 
% Tiivistelmän voi suomen/ruotsin lisäksi kirjoittaa halutessaan
% myös englanniksi. Eli tiivistelmiä tulee yksi tai kaksi kpl.
%
% "\MUUTTUJA"-kohdat luetaan aaltosci_t.sty:ä varten.

\author{Onni Miettinen}

% Otsikko nimiölehdelle. Yleensä sama kuin seuraavana oleva \TITLE, 
% mutta jos nimiölehdellä tarvetta "kaksiosaiselle" kaksiriviselle
\title{Detecting multilinear monomials with algebraic fingerprints}

% Otsikko tiivistelmään. Jos lisäksi engl. tiivistelmä, niin viimeisin:
\TITLE{Detecting multilinear monomials with algebraic fingerprints}

\ENTITLE{Detecting multilinear monomials with algebraic fingerprints}

% suomi:
\DEPT{Tietotekniikan laitos}               
% englanti:
\ENDEPT{Department of Computer Science}    

% Vuosi ja päivämäärä, jolloin työ on jätetty tarkistettavaksi.
\YEAR{2023}
%\DATE{31. helmikuuta 2011}
\ENDATE{\today}

% Kurssin vastuuopettaja ja työsi ohjaaja(t)
\SUPERVISOR{Prof. Eero Hyvönen}
\INSTRUCTOR{D.Sc. Augusto Modanese}
%\INSTRUCTOR{Ohjaajantitteli Sinun Ohjaajasi, ToinenTitt Matti Meikäläinen}
% DI       // på svenska DI diplomingenjör
% TkL      // TkL teknologie licentiat
% TkT      // TkD teknologie doctor
% Dosentti Dos. // Doc. Docent
% Professori Prof. // Prof. Professor
% 
% Jos tiivistelmä englanniksi, niin:
\ENSUPERVISOR{Prof. Eero Hyvönen}
\ENINSTRUCTOR{D.Sc. Augusto Modanese}
% M.Sc. (Tech)  // M.Sc. (Eng)
% Lic.Sc. (Tech)
% D.Sc. (Tech)   // FT filosofian tohtori, PhD Doctor of Philosophy
% Docent
% Professor

% Kirjoita tänne HOPS:ssa vahvistettu pääaineesi.
% Pääainekoodit TIK-opinto-oppaasta.

\PAAAINE{Tietotekniikka}
\ENPAAAINE{Computer Science}

\CODE{SCI3027}

% Avainsanat tiivistelmään. Tarvittaessa myös englanniksi:

\KEYWORDS{multilineaarinen, monomi, havaitseminen, algebrallinen, sormenjälki, parametrisoitu, 
kombinatorinen}
\ENKEYWORDS{multilinear, monomial, detection, algebraic, fingerprints, fingerprinting, 
parameterized, combinatorial}

% Tiivistelmään tulee opinnäytteen sivumäärä.
% Kirjoita lopulliset sivumäärät käsin tai kokeile koodia. 
%
% Ohje 29.8.2011 kirjaston henkilökunnalta:
%   - yhteissivumäärä nimiölehdeltä ihan loppuun
%   - "kaikkien yksinkertaisin ja yksiselitteisin tapa"
%
% VANHA // Ohje 14.11.2006, luku 4.2.5:
% VANHA // - sivumäärä = tekstiosan (alkaen johdantoluvusta) ja 
% VANHA //  lähdeluettelon sivumäärä, esim. "20"
% VANHA // - jos liitteet, niin edellisen lisäksi liitteiden sivumäärä,
% VANHA //  tyyli "20 + 5", jossa 5 sivua liitteitä 
% VANHA // - HUOM! Tässä oletuksena sivunumerointi alkaa nimiölehdestä 
% VANHA //  sivunumerolla 1. %   Toisin sanoen, viimeisen lähdeluettelosivun 
% VANHA //  sivunumero EI ole sivujen määrä vaan se pitää laskea tähän käsin

\PAGES{\pageref{pages:refs}}
%\PAGES{23}  % kaikki sivut laskettuna nimiölehdestä lähdeluettelon tai 
             % mahdollisten liitteiden loppuun. Tässä 23 sivua

%\thispagestyle{empty}  % nimiölehdellä ei ole sivunumerointia; tyylin mukaan ei tehdäkään?!

\maketitle             % tehdään nimiölehti

% -------------- Tiivistelmä / abstract -------------------------------
% Lisää abstrakti kandikielellä (ja halutessasi lisäksi englanniksi).

% Edelleen sivunumerointiin. Eräs ohje käskee aloittaa sivunumeroiden
% laskemisen nimiösivulta kuitenkin niin, että sille ei numeroa merkitä
% (Kauranen, luku 5.2.2). Näin ollen ensimmäisen tiivistelmän sivunumero
% on 2. \maketitle komento jotenkin kadottaa sivunumeronsa.
\setcounter{page}{2}    % sivunumeroksi tulee 2
% Tiivistelmät tehdään viimeiseksi. 
%
% Tiivistelmä kirjoitetaan käytetyllä kielellä (JOKO suomi TAI ruotsi)
% ja HALUTESSASI myös samansisältöisenä englanniksi.
%
% Avainsanojen lista pitää merkitä main.tex-tiedoston kohtaan \KEYWORDS.
\begin{enabstract}
  This thesis studies how parameterized combinatorial problems can be 
  solved %by detecting a $k$-multilinear monomial in a multivariate polynomial 
  by algebraic means. More specifically, the thesis focuses on the technique 
  of algebraic fingerprinting by Koutis and Williams. The thesis 
  overviews the algebra behind the technique and discusses its limits 
  as a general framework for parameterized problems. 
  Finally, the thesis collects some ideas 
  for improving the framework or even surpassing the lower limit of the 
  general framework.

  Combinatorial problems can be reduced to instances of an algebraic problem 
  of detecting a multilinear monomial in a multivariate polynomial. 
  With algebraic fingerprints, a multilinear monomial can be detected 
  in time \bigO{2^k \cdot \poly(n)}, where $n$ is the number of variables in the 
  multivariate $k$-degree polynomial. This gives a general framework for solving parameterized 
  combinatorial problems efficiently. 
  Currently, this technique underlies fastest algorithms 
  for many parameterized combinatorial problems. 

  Moreover, \bigOmega{2^k \cdot \poly(n)} is the lower limit of the technique: 
  multilinear monomials cannot be detected faster 
  with the general framework of algebraic fingerprints. 
  However, this framework solves the very general multilinear monomial 
  detection problem; if the underlying combinatorics of the problem are well understood 
  and taken use of in the reduction to multilinear monomial detection, 
  a much faster algorithm may be 
  found for a specific problem. Indeed, e.g. 
  Björklund found an algorithm that 
  finds a Hamiltonian path in time \bigO{1.657^n \cdot \poly(n)} by 
  using techniques from algebraic fingerprinting. Furthermore, 
  it seems that understanding and making use of the ideas behind algebraic fingerprints 
  is the key to improving from the general framework.
%
%Tiivistelmätekstiä tähän (\languagename). Huomaa, että tiivistelmä tehdään %vasta kun koko työ on muuten kirjoitettu.
\end{enabstract}

\newpage
% Tiivistelmät tehdään viimeiseksi. 
%
% Tiivistelmä kirjoitetaan käytetyllä kielellä (JOKO suomi TAI ruotsi)
% ja HALUTESSASI myös samansisältöisenä englanniksi.
%
% Avainsanojen lista pitää merkitä main.tex-tiedoston kohtaan \KEYWORDS.

\begin{fiabstract}
  Tässä työssä tutkitaan, kuinka parametrisoituja kombinatorisia ongelmia voidaan 
  ratkaista %löytämällä $k$-asteinen multilineaarinen monomi monimuuttujaisesta polynomista 
  algebrallisin keinoin. Kandidaatintyö keskittyy tarkemmin 
  Williamsin ja Koutisin algebrallisia sormenjälkiä hyödyntävään 
  tekniikkaan, jolla voidaan havaita multilineaarinen monomi tehokkaasti. 
  Työ tutkii tekniikan taustalla olevaa algebraa ja sen mahdollisia rajoja,  
  kun sitä tarkastellaan yleisenä kehyksenä parametrisoitujen ongelmien ratkaisemiseksi. 
  Viimeiseksi pohditaan keinoja, joilla tätä kehystä voitaisiin kehittää tai jopa 
  ohittaa sen antama raja algoritmien tehokkuudelle. 

  Kombinatoriset ongelmat voidaan redusoida algebralliseen muotoon, jossa 
  pyritään havaitsemaan multilineaarinen monomi monimuuttujaisesta polynomista. 
  Algebrallisten sormenjälkien avulla multilineaarinen monomi 
  voidaan havaita ajassa \bigO{2^k \cdot \poly(n)}, 
  jossa $n$ on $k$-asteisen polynomin muuttujien lukumäärä. Tämä johtaa yleiseen kehykseen, 
  jolla parametrisoituja kombinatorisia ongelmia voidaan ratkaista. 
  Tällä hetkellä algebrallisten sormenjälkien 
  tekniikka pohjaakin nopeimpia algoritmeja monille parametrisoiduille 
  kombinatorisille ongelmille.

  \bigOmega{2^k \cdot \poly(n)} on kuitenkin 
  kehyksen alaraja: algebrallisten sormenjälkien 
  avulla ei voida yleisesti havaita multilineaarisia monomeja nopeammin. Toisaalta 
  kehys kohdistuu hyvin yleismuotoiseen multilineaaristen monomien havaitsemisongelmaan; 
  jos hyödynnetään ongelmakohtaisia kombinatorisia ominaisuuksia, 
  voidaan löytää algoritmi, joka ratkaisee ongelman nopeammin. 
  Muun muassa Björklund kehitti 
  algebrallisten sormenjälkien antamien ideoiden avulla algoritmin, joka 
  löytää Hamiltonisen polun ajassa \bigO{1.657^n \cdot \poly(n)}. 
  Näyttääkin siltä, että algebrallisten sormenjälkien syvä ymmärrys sekä ongelmakohtainen 
  hyödyntäminen on avain nopeammille algoritmeille ja kehitykselle yleisestä algebrallisesta 
  kehyksestä.
%
%Tiivistelmätekstiä tähän (\languagename). Huomaa, että tiivistelmä tehdään %vasta kun koko työ on muuten kirjoitettu.

\end{fiabstract}

\newpage                       % pakota sivunvaihto

% -------------- Sisällysluettelo / TOC -------------------------------

\tableofcontents

\label{pages:prelude}
\clearpage                     % kappale loppuu, loput kelluvat tänne, sivunv.
%\newpage

% -------------- Symboli- ja lyhenneluettelo -------------------------
% Lyhenteet, termit ja symbolit.
% Suositus: Käytä vasta kun paljon symboleja tai lyhenteitä.
%
%% -------------- Symbolit ja lyhenteet --------------
%
% Suomen kielen lehtorin suositus: vasta kun noin 10-20 symbolia
% tai lyhennettä, niin käytä vasta sitten.
%
% Tämä voi puuttuakin. Toisaalta jos käytät paljon akronyymejä,
% niin ne kannattaa esitellä ensimmäisen kerran niitä käytettäessä.
% Muissa tapauksissa lukija voi helposti tarkistaa sen tästä
% luettelosta. Esim. "Automaattinen tietojenkäsittely (ATK) mahdollistaa..."
% "... ATK on ..."

\addcontentsline{toc}{section}{Käytetyt symbolit ja lyhenteet}

\section*{Käytetyt symbolit ja lyhenteet}
%?? Käytetyt lyhenteet ja termit ??
%?? Käytetyt lyhenteet / termit / symbolit ??
%\section*{Abbreviations and Acronyms}

\begin{center}
\begin{tabular}{p{0.2\textwidth}p{0.65\textwidth}}
3GPP  & 3rd Generation Partnership Project; Kolmannen sukupolven 
matkapuhelupalvelu \\ 
ESP & Encapsulating Security Payload; Yksi IPsec-tietoturvaprotokolla \\ 
$\Omega_i$ & hilavitkuttimen kulmataajuus \\
$\mathbf{m}_{ic}$ & hilavitkutinjärjestelmän $i$ painokertoimet \\
\end{tabular}
\end{center}

\vspace{10mm}

Tähän voidaan listata kaikki työssä käytetyt lyhenteet. Lyhenteistä
annetaan selityksenä sekä alkukielinen termi kokonaisuudessaan
(esim. englanninkielinen lyhenne avattuna sanoiksi) että sama
suomeksi. Jos suoraa käännöstä ei ole tai sellaisesta on vaikea saada
sujuvaa, voi käännöksen sijaan antaa selityksen siitä, mitä kyseinen
käsite tarkoittaa. Jos lyhenteitä ei esiinny työssä paljon, ei tätä
osiota tarvita ollenkaan. Yleensä luettelo tehdään, kun lyhenteitä on
10--20 tai enemmän. Vaikka lyhenteet annettaisiinkin tässä
keskitetysti, ne pitää silti avata sekä suomeksi että alkukielellä
myös itse tekstissä, kun ne esiintyvät siellä ensi kertaa.  Käytetyt
lyhenteet -osion voi nimetä myös ``Käytetyt lyhenteet ja termit'', jos
luettelossa on sekä lyhenteitä että muuta käsitteenmäärittelyä.

\textbf{TIK.kand suositus: Lisää lyhenne- tai symbolisivu, kun se
  näyttää luontevalta ja järkevältä. (Käytä vasta kun lyhenteitä yli 10.)}

%Jos tarvitset useampisivuista taulukkoa, kannattanee käyttää 
%esim. \verb!supertabular*!-ympäristöä, josta on kommentoitu esimerkki
%toisaalla tekstiä.


 
%\clearpage                     % luku loppuu, loput kelluvat tänne
\newpage

% -------------- Kuvat ja taulukot ------------------------------------
% Kirjoissa (väitöskirja) on usein tässä kuvien ja taulukoiden listaus.
% Suositus: Ei kandityöhön.

% -------------- Alkusanat --------------------------------------------
% Suositus: ÄLÄ käytä kandidaatintyössä. Jos käytät, niin omalle 
% sivulleen käyttäen tarvittaessa \newpage
%
%% --------------- Alkusanat -------------------------------------------
%
% Suositus: Älä käytä kandidaatintyössä.
%

\addcontentsline{toc}{section}{Alkusanat}

\section*{Alkusanat}
%\section*{Förord}
%\section*{Acknowledgements}

Alkusanoissa voi kiittää tahoja, jotka ovat merkittävästi edistäneet
työn valmistumista. Tällaisia voivat olla esimerkiksi yritys, jonka
tietokantoja, kontakteja tai välineistöä olet saanut käyttöösi,
haastatellut henkilöt, ohjaajasi tai muut opettajat ja myös
henkilökohtaiset kontaktisi, joiden tuki on ollut korvaamatonta työn
kirjoitusvaiheessa. Alkusanat jätetään tyypillisesti pois
kandidaatintyöstä, joka on laajuudeltaan vielä niin suppea, ettei
kiiteltäviä tahoja luontevasti ole.

\textbf{TIK.kand suositus: Älä käytä tällaista lukua.}

\vskip 10mm
Espoossa 31. helmikuuta 2011
\vskip 15mm
Teemu Teekkari


%\clearpage                     % luku loppuu, loput kelluvat tänne
%\newpage                       % pakota sivunvaihto
%
%SH: Alkusanoissa voi kiittää tahoja, jotka ovat merkittävästi edistäneet
% työn valmistumista. Tällaisia voivat olla esimerkiksi yritys, jonka
% tietokantoja, kontakteja tai välineistöä olet saanut käyttöösi,
% haastatellut henkilöt, ohjaajasi tai muut opettajat ja myös
% henkilökohtaiset kontaktisi, joiden tuki on ollut korvaamatonta työn
% kirjoitusvaiheessa. Alkusanat jätetään tyypillisesti pois
% kandidaatintyöstä, joka on laajuudeltaan vielä niin suppea, ettei
% kiiteltäviä tahoja luontevasti ole.

% ---------------------------------------------------------------------
% -------------- Tekstiosa alkaa --------------------------------------
% ---------------------------------------------------------------------

% Muutetaan tarvittaessa ala- ja ylätunnisteet
%\pagestyle{headings}          % headeriin lisätietoja
%\pagestyle{fancyheadings}     % headeriin lisätietoja
%\pagestyle{plain}             % ei header, footer: sivunumero

% Sivunumerointi, jos käytetty 'roman' aiemmin
% \pagenumbering{arabic}        % 1,2,3, samalla alustaa laskurin ykköseksi
% \thispagestyle{empty}         % pyydetty ensimmäinen tekstisivu tyhjäksi

% input-komento upottaa tiedoston 
% --------------------------------------------------------------------

\section{Introduction}

In recent years, there have been rapid advances in the algorithms for combinatorial problems. 
This has been greatly sparked by the development in algebraic techniques for solving the 
multilinear monomial detection problem, i.e., finding whether a multivariate polynomial contains a multilinear monomial, 
first introduced by \citeauthor{Koutis05} in \cite{Koutis05}, where the set packing problem is
reduced to multilinear monomial detection.

Namely, the technique of algebraic fingeprinting, first introduced in
\cite{Koutis08} and further developed in \cite{Williams09}, 
has found great success for many combinatorial problems. 
For example, with algebraic fingerprinting, the $k$-path problem 
that previously could be solved in \bigOstar{4^{k}} time by \citeauthor{Chen07} in \cite{Chen07}, 
could be solved in \bigOstar{2^{3k/2}} time in \cite{Koutis08}. 
\amnote{What is the meaning of \bigOstar{}?}
This result was quickly improved in \cite{Williams09}, where an \bigOstar{2^k} algorithm was given.

\amnote*[inline,nomargin]{This technique was further developed by
\citeauthor{Björklund14} \cite{Björklund14}, who showed ...}{
Of course, this technique was further developed, and soon after in
\cite{Björklund14} Björklund et al. showed an algorithm 
}
\amnote{Note use of \texttt{\textbackslash citeauthor}}
that solved the Hamiltonian problem (Hamiltonicity), i.e., finding whether a given graph contains a simple path that visits 
every vertex, in \bigOstar{1.657^n} time. Soon enough, for $k$-path, an \bigOstar{1.66^k} algorithm was found \cite{Björklund17}. 
\amnote{Did Björklund really develop the technique further? or did he only show
how to apply it to another problem?}
The fastest algorithms for Hamiltonicity before this ran in
\amnote*{$n$ or $k$?}{ \bigOstar{2^n} }
and were known since 1962 \cite{HelKar62}, \cite{Bellman62}.
This was a significant improvement on a problem that had seen no progress in nearly fifty years.

\subsection{Research goals and thesis structure}
\amnote[inline,nomargin]{This section can be shortened to one paragraph}

The goals of this thesis are to find out how multilinear monomial detection is relevant in 
combinatorial problems, and how algebraic fingerprints can be utilized to design faster 
algorithms for problems that use multilinear monomial detection. Also, interesting ideas 
regarding algebraic fingerprints are explored for.

Multilinear monomial detection 
is a fundamental problem, since many important combinatorial problems can be reduced into it 
via a problem specific algebraization. Thus, faster algorithms and new ideas for multilinear monomial detection 
are important.

Multilinear monomial detection is essentially searching for solutions among non-solutions, 
both of which are encoded as monomials in a polynomial. 
The technique of algebraic fingerprinting is present in multilinear monomial detection. 
With algebraic fingerprints, unwanted cancelation of solution monomials due to the characteristic of a field can be prevented. 
Moreover, algebraic fingerprints can be used to cancel non-solution monomials by abusing the characteristic.

\amnote*{No need for this since you already have ToC}{%
In the next subsections, the thesis discusses algebraization and reduction into multilinear monomial detection. 
The section 2 covers preliminaries. In section 3, the thesis discusses general multilinear monomial detection. 
In section 4, some problem specific instances of multilinear monomial detection are given, and clever 
utilizations of algebraic fingerprints are shown. Section 5 concludes the
thesis.
}

\subsection{Algebrization of combinatorial problems}

A combinatorial problem asks whether a given finite set of objects satisfies some given constraints. 
For example, the $k$-path problem asks for, given a finite set of vertices and edges, 
a simple path of $k$ vertices. The solutions and non-solutions (solution space) to combinatorial problems can be thought of as 
combinations of the given objects.
\amnote*{Probably one can drop the vertices and think only about edges?}{%
The solution space for the $k$-path problem
consists of combinations of $k$ vertices and $k-1$ edges. 
}
A non-solution combination would contain duplicate vertices, or edges that
contain vertices outside the combination.

\amnote[inline,nomargin]{This sounds like a very simple problem... But I guess
the catch is that we want an algorithm with complexity depending only on $k$?}

Algebraization is reducing a given problem into an algebraic form,
\amnote*[inline,nomargin]{that is}{i.e.}, a
question regarding some algebraic property of some algebraic entity. 
In an algebraization of a combinatorial problem, the algebraic entity can be constructed from algebraic elements defined from the 
set of objects given as an input. The motivation behind the construction is some algebraic property that, 
when satisfied, gives a solution to the problem.

\amnote{Somewhere we need to reference the reader to Sect.~2 for non-familiar
terms like multivariate, algebras, etc.}
In \cite{Valiant92}, it was observed that multivariate polynomials in certain algebras have natural combinatorial interpretations. 
Utilizing this idea, \cite{Koutis05} managed to reduce a combinatorial problem
into an algebraic form, that is, multilinear monomial detection. 
First, we introduce multiple variables that 
correspond to elements from the set of objects given as input. 
Then, we construct an arithmetic circuit representing a multivariate polynomial, such that it 
encodes all solutions and non-solutions as multivariate monomials, 
with multilinear monomials corresponding to solutions. 
Thus, the task of finding a satisfying combination to the combinatorial problem has been reduced to 
finding a multilinear monomial from the multivariate polynomial.
It follows that a decision problem is answered by 
the existence of a multilinear monomial, and a counting problem by the number of multilinear monomials.

Appropriate definitions for the variables are problem-specific. In the following section, this thesis gives a 
reduction into multilinear monomial detection, shown in \cite{KouWil15}, 
for the $k$-3D matching problem. Another simple example can be found, for the set packing problem, in \cite{Koutis05}.

\amnote[inline,nomargin]{I would merge the two sections and use the $k$-3D
matching as an example to the explanation above}

\subsection{Reducing $k$-3D matching into multilinear monomial detection}
\label{sect:reduction_example}

The $k$-3D matching problem is defined as follows:

\begin{problem}
  \problemtitle{$k$-\textsc{3D matching}}
  \probleminput{Three disjoint sets $A$, $B$ and $C$, and a set of triples $T\subset A\times B\times C$.}
  \problemquestion{Is there a subset $M\subseteq T$, such that $\abs{M} = k$ and 
  $\forall m \in M$: None of the elements in $m$ appear in
  \amnote*[inline,nomargin]{$M \setminus \{m\}$}{$M\backslash \{m\}$}}
\end{problem}
\amnote{Use \texttt{\textbackslash setminus} instead of \texttt{\textbackslash
backslash}}
\amnote{$T = A \times B \times C$ not allowed?}
\amnote{Either convert $\forall$ into text or what follows into maths}

We begin by defining new variables corresponding to the elements in $A$, $B$ and $C$, 
labeled as $a_i$, $b_j$ and $c_k$, respectively, where $i\in [\abs{A}]$, $j\in
[\abs{B}]$ and $k\in [\abs{C}]$. 
\amnote{Def. of $\abs{\cdot}$?}

For every triple $t \in T$, we define a multilinear monomial $x$ that is a
product of the elements in $t$. 
\amnote{Over what object are we doing multiplication?}
We introduce a set $X$ that satisfies the following:
%\begin{center}$\forall x \in X$: $x = abc$ : $(a, b, c) \in T$.\end{center}
\[
\forall x \in X: x = abc : (a, b, c) \in T.
\]
\amnote{Use \textbackslash[ ... \textbackslash] for display math}
\amnote[inline,nomargin]{Probably meant $\forall x \in X, x = abc ...$?}

Next, we define multivariate polynomials $P_1$ and $P_k$ as follows:
\begin{center}$P_1 = \displaystyle \sum_{X}$ ,   $P_k = P_1^k$.\end{center}
\amnote{Meaning of $\sum_X$?}

Following this construction, we observe that $P_k$, when expanded into a sum of multivariate monomials, 
contains a multilinear term if and only if the original $k$-3D matching instance can be answered in the positive. 
Furthermore, every multilinear monomial in the expanded $P_k$ corresponds to a solution to the problem, and 
the solutions can be directly found from the variables in the multilinear monomial. Thus, 
a successful reduction into multilinear monomial detection has been given for the $k$-3D mapping.

An example instance of $k$-3D matching with this exact algebraization can be found in \cite{KouWil15}. 
TODO: show the example here
\clearpage
\section{Related works} %TODO: maybe change headline?
\label{sect:related_works}

TODO: go through publications that utilize this algebraic fingerprinting technique, 
and problems that reduce to multilinear detection
\clearpage

\section{Preliminaries}
\label{sect:prelims}

It is necessary to recall basic algebraic concepts 
before further discussing multilinear monomial detection and algebraic fingerprinting. 
\cref{sect:prelims_algebra} gives definitions for a group, ring and field, 
and some useful concepts regarding them. 
\cref{sect:prelims_other} goes through other notation and terminology used 
throughout the thesis.

\subsection{Groups, rings and fields} %polynomial rings, group algebra, char of field
\label{sect:prelims_algebra}

A group \amnote*{There is also \texttt{\textbackslash mathbf}}{$\textbf G$}
is a tuple $(G, +)$, where $G$ is a set of elements,
\amnote{See TeX source}
%$+$ : $G \times G \longrightarrow G$
$+ \colon G \times G \to G$
is a binary operation closed under 
the elements in $G$, $+$ is associative, every element $g\in G$ has an inverse $g^{-1}\in G$, and $G$ contains 
an identity element $e$ such that $g + e = g$, $g + g^{-1} = e$ and $e = e^{-1}$. Moreover, $\textbf G$ is called $Abelian$ if 
$+$ is also commutative.
\amnote{I think $e = e^{-1}$ follows from the other requirements (but that is
not so important)}

A ring $\textbf R$ is a tuple
\amnote*[inline,nomargin]{$(R,+,\cdot)$}{$(R, \cdot )$},
where $R = (G, +)$ is an Abelian group, $\cdot$ : $G \times G \longrightarrow G$ 
is a binary operation closed under $G$. We call the binary operations $+$ and
$\cdot$
%as
addition and multiplication, respectively. 
Note, that from here on we use $R$ as the set of elements defined for $\textbf R$. 
In general, a bold typeface \textbf X represents a group, ring or field and $X$ its set of elements. 
$R$ must contain a multiplicative identity $\textbf 1 \in R$ such that $\forall a \in R$: $a \cdot \textbf 1 = a$. 
We notate the additive identity $e$ required for the group as $\textbf 0$ from here on. 
Observe, that for any $R \neq \{\textbf 0\}$, $\textbf 1 \neq \textbf 0$.  
Left and right distributive laws hold for rings, i.e., 
\begin{center}
  $\forall a, b, c \in R$: $a \cdot (b + c) = (a \cdot b) + (a \cdot c) \land (b + c) \cdot a = (b \cdot a) + (c \cdot a)$.
\end{center}
$u \in R$ is called $unit$ if it holds that $\exists v \in R$: $u \cdot v = v \cdot u = \textbf 1$, 
i.e., it has a multiplicative inverse $v \in R$.

A field $\textbf F = (F, +, \cdot)$ is defined with the following conditions:
\begin{itemize}
  \item $(F, +)$ is an Abelian group
  \item $(F\backslash \{\textbf 0\}, \cdot )$ is an Abelian group
  \item Left and right distributive laws hold for \textbf F
\end{itemize}

Equivalently, a ring is a field if every non-zero element is unit, $\textbf 1 \neq \textbf 0$, and multiplication is commutative. 
The $characteristic$ of a field \textbf F is defined as follows:
\begin{equation}
  char(\textbf F) =
    \begin{cases}
      min\{n \in \N : n \cdot \textbf 1 = \textbf 0 \}\\
      0 & \text{if such $n$ does not exist}\\
    \end{cases}       
\end{equation}


Note, that a field \textbf F with 
characteristic 2 satisfies the following:
\begin{center}
  $\forall u \in F$: $u + u = u \cdot (\textbf 1 + \textbf 1) = u \cdot \textbf 0 = \textbf 0$
\end{center}

TODO: group algebra, polynomial ring, linear dependency, "cancelation due to characteristic", 
Galois field, "multiplicative group"\nl
%TODO: Cygan, p. 337

\subsection{Notation and other terminology} 
\label{sect:prelims_other}

TODO: create a table or like, list terms: 
multilinearity, multivariety, sum of monomials form \& generating form (arithmetic circuit) of polynomial, 
degree of multivariate monomial, $\mathcal{O}$, $\Theta$, FPT, $\mathcal{O}$*, determinism \& non-determinism [or use 'Monte Carlo' :)], 
Schwartz-Zippel lemma

\amnote[inline,nomargin]{Standard $\mathcal{O}$ and $\Theta$ notation should be
known}
\clearpage
\section{General multilinear monomial detection}

The detection of multilinear monomials in a multivariate polynomial is a fundamental problem, 
since many important problems can be reduced to it [TODO: quick examples (just refs?)]. 
Therefore, any progress in general multilinear monomial detection directly implies 
faster algorithms
\amnote*{These problems should \enquote{live} in their own subsection, then you
can just call them \enquote{problems from Sect.~X.Y}}{%
for all problems, that are reduced to and solved with general
multilinear monomial detection. 
} TODO: relate to section 2: related works

In this section, we first define the parameterized multilinear monomial detection problem, 
and give some non-algebraic background on solving it. Then in Section \ref{sect:algebraic_fingerprinting}, 
we discuss the algebraic ideas by \citeauthor{KouWil15} \cite{Koutis08, Williams09, KouWil15}. 
Section \ref{sect:algebraic_fingerprinting} is long, but the overall idea as a 
general framework for parameterized problems is 
summarized and discussed in Section \ref{sect:algebraic_framework}. 
Finally, we briefly discuss finding the solution when its 
existence is detected in Section \ref{sect:finding_the_solution}.

\subsection{Problem definition}

The general, parameterized multilinear monomial detection problem is defined as follows: 

\amnote{Arithmetic circuit has bounded or unbounded fan-in / fan-out?}
\begin{problem}
  \problemtitle{$k$-\textsc{multilinear monomial detection}}
  \probleminput{A commutative arithmetic circuit $A$ over a set of variables $X$ representing a polynomial $P(X)$.}
  \problemquestion{Does the polynomial $P(X)$ extended as a sum of monomials 
  contain a multilinear monomial of degree $k$?}
\end{problem}

Clearly, an upper bound for solving the problem is given by a 
naive expansion of $A$ into $P(X)$ and evaluation of $P(X)$.
However, this is not optimal: an $N$-degree polynomial will have $2^{\Theta(N)}$ 
possible monomials, and most problems that can use this algebraization, 
have been solved with faster algorithms. %(TODO: quick example?) 
This motivates the detection of multilinear monomials 
without fully expanding $A$ into a sum of monomials.

Since only multilinear terms are important in $P(X)$, 
any squared variable can be instantly discarded as soons as it is formed in $A$. 
This can be achieved with dynamic programming to create a polynomial $P'(X)$ that 
only contains multilinear monomials. Since there are $2^N$ multilinear monomials in $P'(X)$ with 
$N$ variables, this method results in a
\amnote*{but still exponential}{faster} algorithm than with naive expansion.

%TODO: Maybe rewrite the algebra here 
%(first quickly in English, then in maths, and maybe label the mapping functions)
However, the underlying problems are usually FPT. This implies that scaling exponentially 
with the number of variables is far from optimal. In order for the complexity of the algorithm to scale 
with the parameter $k$, we can reduce the number of variables by mapping $X$ into $Y$, where 
$\abs{X} \geq \abs{Y}$ and
\amnote*{Def?}{$\abs{Y} \propto k$}, and dynamically evaluate $P(Y)$ instead of
$P(X)$. 
\amnote{$P$ is unchanged, so it still expects $\abs{X}$ arguments...?}

However, since $\abs{X} \geq \abs{Y}$, a multilinear monomial in $P(X)$ may not be multilinear in $P(Y)$. 
For an algorithm to detect a multilinear monomial, it is necessary to use different mappings 
from $X$ into $Y$ until a multilinear monomial survives the mapping. However, the probability 
that any given multilinear monomial survives this mapping is around $e^{-\abs{Y}}$ \cite{KouWil15}.
This implies that \bigO{e^{\abs{Y}}} random mappings must be tried for a
multilinear monomial to survive with a reasonable constant probability. Thus, a
$k$-multilinear monomial is detected with a 
randomized algorithm in \bigOstar{(2e)^{\abs{Y}}} time, where $\abs{Y} \propto k$.

This is essentially the idea behind color coding introduced by \citeauthor{Alon95} \cite{Alon95}. 
Although here given in this algebraic form for the $k$-multilinear monomial detection, 
color coding is combinatorial, and does not rely on algebraic techniques. 
However, since $k$-multilinear monomial detection is a purely algebraic problem, it is reasonable to 
conjecture that there is an algebraic method for solving it.

\omnote{I think this should maybe be in the introduction section rather than here?}
\amnote{I agree}
Indeed, a faster algebraic method exists; the technique of algebraic fingerprinting 
first introduced by \citeauthor{Koutis08} \cite{Koutis08} and 
further developed by \citeauthor{Williams09} \cite{Williams09} 
solves $k$-multilinear monomial detection in \bigOstar{2^k} time. 
Since the technique focuses on the abstract $k$-multilinear monomial detection, to which 
most parameterized problems can be reduced to, it gives a general framework for 
solving parameterized problems \cite{KouWil15}.

\subsection{Algebraic fingerprinting}
\label{sect:algebraic_fingerprinting}

The idea of discarding squared variables as soon as they are formed in $A$ 
can be expressed algebraically: any squared variable should be identical to
zero. 
\amnote{Quotient ring}
\begin{equation}
  \label{eq:squared_to_zero}
\forall x \in X: x^2 = \mathbf{0}
\end{equation}
This implies that any non-multilinear monomial will evaluate to zero in $P(X)$, since 
any non-multilinear monomial $q$, assuming commutativity, can be written as $x^2q'$ 
for some $x \in X$, and $x^2q' = \mathbf{0}q = \mathbf{0}$. 
Therefore, $P(X)$ will identically evaluate to zero if there are no multilinear monomials, 
i.e., there exist no solutions to the original problem.

This is the basis of algebraic multilinear monomial detection introduced by 
\citeauthor{Koutis08} \cite{Koutis08}, or later referred to as \emph{algebraic fingerprinting} \cite{KouWil15}: 
we evaluate $P(X)$ over some algebra $\mathbf{G}$, and detect a multilinear monomial from 
the value returned by this evaluation. Ideally, with any $\gamma \colon X \to G$, $P(X')$ representing 
$P(X)$ over $\mathbf{G}$ by $\gamma$ and $w \neq \mathbf{0}$, 

\begin{equation}
  \label{eq:polynomial_identity}
  P(X') =
    \begin{cases}
      \mathbf{0} & \text{if no multilinear monomials exist}\\
      w & \text{otherwise}\\
    \end{cases}       
\end{equation}

In the following subsections, we specify an appropriate algebra such that 
(\ref{eq:polynomial_identity}) is met, as well as some requirements for efficiency. 
Then, we discuss the works of \citeauthor{Koutis08} and \citeauthor{Williams09} \cite{Koutis08, Williams09}, 
and see how these specifications were implemented.

This thesis refers to the general framework \cite{KouWil09, KouWil15} by these 
authors as algebraic fingerprinting. However, algebraic fingerprinting can also 
be used to refer to the idea behind solving a problem 
with multilinear monomials canceling out due to characteristic, 
which is discussed here in Section \ref{sect:fingerprints}.

\subsubsection{Specifications for the algebra}

We have arrived at an important task for multilinear monomial detection: 
find a field $\mathbf{G}$ for 
the assignment $\gamma \colon X \to \mathbf{G}$ to meet 
(\ref{eq:squared_to_zero}) and thus the first equality in (\ref{eq:polynomial_identity}). 
We specify for fields since rings are not enough for multilinear monomial detection; 
$\mathbf{G}$ should have commutative multiplication in order for (\ref{eq:squared_to_zero}) 
to be effective. The ordering of indeterminates in monomials is generally unknown, 
since the specific algebrization into multilinear monomial detection 
is abstracted away. %Specifying for commutativity also makes it significantly 
%easier to algebrize problems into arithmetic circuits. 

For the second equality in (\ref{eq:polynomial_identity}), it is necessary that multilinear 
monomials in $P(X)$ can map to multilinear monomials in $P(X')$, i.e., it must be that 
$k \leq \abs{G}$, where $k$ is the degree of the monomials. 
Moreover, multilinear monomials 
should not evaluate to $\mathbf{0}$ over $\mathbf{G}$, and more specifically, 
$w$ should not be identical to $\mathbf{0}$.

These are the necessary specifications for the field $\mathbf{G}$ for algebraic multilinear monomial detection. 
However, this algebraic detection must be faster than color coding for it to be useful. 
Therefore, further requirements are necessary: (a) the binary operations of $\mathbf{G}$ 
must be fast for a fast evaluation of $P(X')$, and (b) multilinear monomials must 
survive the assignment $\gamma$ with a \emph{reasonable} constant probability. 
We may specify a reasonable probability as something around at least 1/4, since that is 
the probability of survival reached in the original work for algebraic fingerprinting \cite{Koutis08}.

Recall that with color coding, multilinear monomials can be detected with a 
randomized algorithm in \bigOstar{(2e)^k} time. Thus for (a), we may specify for 
the evaluation of $P(X')$ to take \bigOstar{2^k} time. The polynomial $P(X)$ 
will have at most $2^k$ multilinear monomials. Therefore, 
it is necessary that the binary operations 
of $\mathbf{G}$ take polynomial time.

With (b), recall that multilinear monomials 
survive color coding with probability $e^{-k}$. 
With algebraic fingerprinting, although not identical to zero, a multilinear monomial 
can still evaluate to zero over $\mathbf{G}$. 
However, if we specify for a constant 
probability of survival, we can reliably decide whether a multilinear monomial exists 
by evaluating $P(X)$ in $\mathbf{G}$ over a constant number of 
randomized assignments $X \to \mathbf{G}$. Relating to color coding, 
this would essentially remove the 
$e^k$ factor in \bigOstar{(2e)^k}.

To restate, we now have to find such a field $\mathbf{G}$ that 
meets the following specifications: 
\begin{itemize}
  \item $\forall g \in \mathbf{G} \colon g^2 = \mathbf{0}$
  \item Operating over $\mathbf{G}$ should be fast, i.e., evaluating 
  $P(X')$ should take \bigOstar{2^k} time.
  \item Multilinear monomials should evaluate to non-zero
  through a random assignment 
  $\gamma \colon X \to \mathbf{G}$ with a reasonable constant probability.
\end{itemize}

In the work that introduced 
this algebraic fingerprinting technique \cite{Koutis08}, 
\citeauthor{Koutis08} used the group algebra $\Z_2[\Z_2^k]$. 
\citeauthor{Williams09} developed the technique further, 
utilizing the algebra $GF(2^{3+log_2(k)})[\Z_2^k]$ \cite{Williams09}. 
Next, we look at these 
group algebras of $\Z_2^k$ for $\mathbf{G}$. 
%TODO: CHECK: and see how \emph{fingerprints} were used to 
%solve a problem with the characteristic of $\Z_2^k$.

\subsubsection{Using group algebras of $\Z_2^k$}

The multiplicative group $\Z_2^k$ consists of $k$-dimensional \{0,1\}-vectors 
with the binary operation defined as component-wise addition modulo 2. 
For example with $k = 3$, 
\[
  \begin{bmatrix} 0 \\ 0 \\ 1 \end{bmatrix} \cdot 
  \begin{bmatrix} 0 \\ 0 \\ 1 \end{bmatrix} =
  \begin{bmatrix} 0 \\ 0 \\ 0 \end{bmatrix} = \mathbf{0} \in \Z_2^3.
\]
Observe that in general, every element in $\Z_2^k$ is its own inverse:
\begin{equation}
  \label{eq:Z_2^k has char 2}
  \forall z \in \Z_2^k \colon z^2 = \mathbf{0}.
\end{equation}
Recall that the elements of a group algebra $\mathbf{F}[\Z_2^k]$ are linear combinations of the form 
\[
  \displaystyle \sum_{v \in \Z_2^k}a_v v,
\]
where $a_v \in \mathbf{F}$. From here on, we note the identity of $\Z_2^k$ as $v_0$, additive and 
multiplicative identities of $\mathbf{F}$ as $\mathbf{0}_F$ and $\mathbf{1}_F$, respectively, and 
use $\mathbf{0}$ and $\mathbf{1}$ for $\mathbf{F}[\Z_2^k]$. Note that $\mathbf{1} = v_0$, and
$\mathbf{0}$ corresponds to the element $\sum_{v \in \Z_2^k}a_v v$, where $a_v = \mathbf{0}_F$.

In \cite{Koutis08}, \citeauthor{Koutis08} assigned $X$ 
to elements of the form $(v_0 + v_i) \in \mathbf{F}[\Z_2^k]$, 
such that for every $x_i \in X$, a random $v_i \in \Z_2^k$ is independently and uniformly 
picked for the assignment 
$x_i \to (v_0 + v_i)$. 
We note the assigned values as $\overbar{X}$, and the resulting polynomial as $P(\overbar{X}) \in \mathbf{F}[\Z_2^k]$. 
\citeauthor{Koutis08} observed that 
due to (\ref{eq:Z_2^k has char 2}), for all $v \in \Z_2^k$ and $(v_0 + v) \in \mathbf{F}[\Z_2^k]$, 
\[
  (v_0 + v)^2 = v_0^2 + v_0v + vv_0 + v^2 = v_0 + v + v + v_0 = 2v_0 + 2v.
\]
This implies that if we pick a field with characteristic 2 for $\mathbf{F}$, 
$\forall v_i \in \Z_2^k \colon (v_0 + v_i)^2 = \mathbf{0}$. Thus, 
non-multilinear monomials in $P(X)$ vanish in $P(\overbar{X})$, and the 
first equation in (\ref{eq:polynomial_identity}) 
will hold.

However, if $\mathbf{F}$ has characteristic 2, the second equation of (\ref{eq:polynomial_identity}) 
does not necessarily hold, and it may be that $w = \mathbf{0}$. Multilinear monomials may 
cancel each other out in $\mathbf{F}[Z_2^k]$, since they may have even leading 
coefficients in $P(X)$. Next, we see how \citeauthor{Koutis08} approached this problem.

%Such $\mathbf{G}$ is given in \cite{Koutis08}, where \citeauthor{Koutis08} uses the group $\Z_2^k$ 
%and a known \cite{Terras99} 
%isomorphic mapping $\rho \colon \Z_2^k \to M^{2^k \times 2^k}$, where $M^{d \times d}$ is a 
%\amnote*{Any group? Specific one?}{$d$-dimensional matrix group with matrix multiplication.}
% Note that for (\ref{eq:squared_to_zero}), 
%\[
%\forall a \in \Z_2^k \setminus \{\mathbf{0}\}, i \in [2^k]: \rho(a)_{i,i} = 0\]and \[
%\rho(\mathbf{0}) = \mathbf{0}.
%\]
\subsubsection{Fingerprints to prevent unwanted cancelation}
\label{sect:fingerprints}

Indeed, if we take the example in Section \ref{sect:reduction_example}, 
we notice that the multilinear monomials have even coefficients, and thus 
would cancel out in $\mathbf{F}[Z_2^k]$ due to characteristic. TODO: 
write the example in the section and then 
add here some   
example polynomial from there that has even coefficients

In general, when $k$ is even, multilinear monomials will have even coefficients. 

To tackle this issue, one idea is to add auxiliary indeterminates, 
called \emph{fingerprints} \cite{KouWil15}, to the monomials 
in order to make them unique. For example, let $S = \{s_1, s_2, \ldots\}$ 
be the set of an appropriate number of fingerprints 
and augment them to (EXAMPLE POLYNOMIAL) as follows: 
(TODO: add fingerprinted polynomial here) 

Then, the algorithm could assign every $a \in X \cup S$ to $\mathbf{F}[\Z_2^k]$. 
With this, non-multilinear monomials will still vanish, but multilinear monomials 
will not identically cancel each other out when $k$ is even. 

However, introducing new indeterminates raises the degree of multilinear monomials. 
Therefore, it increases the probability that variables in a multilinear monomial are 
assigned the same value from $\mathbf{F}[Z_2^k]$, which results in the 
multilinear monomial evaluating to zero with higher probability. 
Raising the dimension of $\Z_2^k$, however, would  
exponentially slow down matrix multiplications which are %matrix mulitplication has a time complexity of \bigO{n^2.31788} \cite{DuanZhouWu22}. 
important for efficiency in the full algebraic framework (see Section \ref{sect:algebraic_framework}).

\citeauthor{Koutis08} approached this problem by assigning fingerprints to a 
different algebra: he used $S \to \Z_2$ and set 
$\mathbf{F} = \Z_2$ \cite{Koutis08}. Note that $\Z_2$ has characteristic 2. With this, 
\citeauthor{Koutis08} essentially assigns multilinear monomials a coefficient 
randomly from $\{\mathbf{0}_F, \mathbf{1}_F\}$. The idea is that a multilinear monomial, 
assigned with the fingerprint $\mathbf{1}_F$,  
survives the cancelation due to characteristic if the canceling pair is assigned $\mathbf{0}_F$. 
With randomized assignments $S \to \Z_2$, there is a constant probability 
(TODO: what is the probability?) that a multilinear 
monomial will have an odd coefficient, and thus survive the assignment \cite{Koutis08}.

In practise, fingerprints can be implemented into the algebrization as follows \cite{Williams09}: 
for every multiplication gate $G_i$ in the arithmetic circuit $A$ for $P(X)$, 
pick a unique $s_i \in S$. Insert a new multiplication gate $\overbar{G_i}$ that takes 
as input $s_i$ and the output of $G_i$. The output of $\overbar{G_i}$ feeds to the 
gates that read the output of $G_i$. We note the new polynomial represented by this circuit 
as $P(X, S)$. Note that here, 
\citeauthor{Koutis08} picked random elements from $\Z_2$ 
instead of picking unique fingerprints $s_i$.

In another perspective, we may look at algebraic fingerprinting as \emph{polynomial identity testing}. 
That is, we compute $P(X, S)$ over assignments $X \to \mathbf{F}[\Z_2^k]$ into $P(\overbar{X}, S)$, i.e., 
compute until the gates $\overbar{G_i}$. 
Imagine we stop here in the circuit before assigning the fingerprints $S$ to some algebra 
and continuing with the multiplication. 
Now, deciding whether a multilinear monomial exists in $P(X)$ is essentially given by 
whether the polynomial $P(\overbar{X}, S) \in \mathbf{F}[\Z_2^k]$ is identical to $\mathbf{0}$, 
i.e., with $\Phi$ representing the family of mappings $S \to \mathbf{F}$, 
\[
  \forall \phi \in \Phi %, \overbar{S} = \{\phi(s) \: | \: s \in S\} 
  \colon P(\overbar{X}, \phi(S)) = \mathbf{0}.
\]

\subsubsection{Polynomial identity testing}

\begin{problem}
  \problemtitle{\textsc{Polynomial identity testing}}
  \probleminput{An arithmetic circuit $C$ that computes the polynomial $Q(S)$.}
  \problemquestion{Is $Q(S)$ identical to the zero polynomial?}
\end{problem}

We frame $P(\overbar{X}, S)$ as $Q(S) \in \mathbf{F}[\Z_2^k]$. 
Here, it is necessary to test whether $Q(S)$ is identical to \emph{zero modulo 2}, 
since we used characteristic 2 to eliminate 
the underlying non-multilinear monomials in $P(\overbar{X}, S)$.

Thus, multilinear monomial detection is reduced via algebraic fingerprinting 
to polynomial identity testing, where a multilinear monomial is detected if 
$Q(S) \neq \mathbf{0}$ over $\mathbf{F}[\Z_2^k]$ with any field $\mathbf{F}$ of characteristic 2. 
We note the family of assignments $S \to \mathbf{F}$ as $\Phi$.

Assume multilinear monomials exist in $P(X)$. 
\citeauthor{Koutis08} achieved to detect a multilinear monomial 
with a probability $(1/4 + 1/{4k})$ by testing whether 
$Q(S) = \mathbf{0}$ over the field $\mathbf{F} = \Z_2$  
with randomized assignments $S \to \Z_2$ \cite{Koutis08}. 
\citeauthor{Williams09} developed the technique of algebraic fingerprinting 
further by observing that due to the Schwartz-Zippel lemma, %(see \cref{sect:prelims}), 
if we raise the order of the field $\mathbf{F}$ 
such that the number of different assignments in $\Phi$ is much larger than 
the number of assignments $\phi \in \Phi$ that have $Q(\phi(S)) = \mathbf{0}$, 
$Q(\delta(S)) \neq \mathbf{0}$ with a high probability over some $\delta \in \Phi$ 
\cite{Williams09}. 

Of course, $\mathbf{F}$ must have characteristic 2. For this, \citeauthor{Williams09} 
used the field $GF(2^{3+log_2k})$ \cite{Williams09}. By Schwartz-Zippel lemma, $Q(S)$ 
evaluates to $\mathbf{0}$ over a random assignment $S \to GF(2^{3+log_2k})$ with probability 
at most $1/2^3$ \cite{Williams09}. For the $k$-path problem, \citeauthor{Koutis08} 
gave a randomized \bigOstar{2^{3k/2}} time algorithm \cite{Koutis08}, 
and \citeauthor{Williams09} developed this into a randomized \bigOstar{2^k} 
time algorithm \cite{Williams09} with the ideas discussed here.

TODO: talk briefly about the other requirement (fast operations), 
results in a \bigOstar{2^k} time algorithm.

In the following section, we see this as a whole and look further into implementing this stuff. 
TODO: rephrase

\subsection{Algebraic fingerprinting as a framework for parameterized problems}
\label{sect:algebraic_framework}

From Section \ref{sect:related_works}, 
it can be seen  
that a fast algorithm for $k$-multilinear monomial detection gives a 
fast algorithm for many parameterized combinatorial problems. Thus, we 
may say that algebraic fingerprinting gives a general framework for 
parameterized combinatorial problems. 

In Section \ref{sect:algebraic_fingerprinting}, we discussed the idea behind 
algebraic fingerprinting: generate a polynomial $P(X)$ from the algebrization of 
a combinatorial problem, and evaluate $P(X)$ over $GF(2^{3+log_2k})[Z_2^k]$ with 
randomized assignments $X \to GF(2^{3+log_2k})[Z_2^k]$ of form 
$x_i \to (v_0 + v_i)$, augmented with scalar multiplications 
by elements randomly chosen from $GF(2^{3+log_2k})$. This gives a  
randomized algorithm for $k$-multilinear monomial detection that runs in \bigOstar{2^k} time.

This section gives a general overlook on algebraic fingerprinting as a 
framework for parameterized problems. First, we discuss the \omnote*{give all the numbers here?}
{time and space complexities} of algebraic fingerprinting. Then, 
we go over an example implementation 
of algebraic fingerprinting for the $k$-path problem. 
Finally, we discuss the limits of 
this technique: the chosen algebra $GF(2^{3+log_2k})[Z_2^k]$ 
is optimal for the general multilinear monomial detection.


\subsubsection{Time and space complexity}
\label{sect:complexity}

TODO: Probably something about the circuits (time and space complexity of operations)

TODO (when talking circuits): arithmetic circuits have addition gates with unbounded fan-in, 
multiplication gates of fan-in two, no scalar multiplication 
(augment scalar multiplication (fingerprints) "later")

\subsubsection{Implementation}

TODO: give an example how to get an algorithm, e.g., for $k$-path?

\subsubsection{Limits of the framework}
\label{sect:limits}

SHORT SECTION

%TODO: give some cons wrt. color coding (difficult derandomization, 
%are we able to add weights for optimization problems?) %(cost-constrained mld?)

TODO: explain the limit in general multilinear detection with this algebraic framework 
(impossible to find better algebra than what is used for the current fastest k-mld algorithm), \cite{KouWil09}

\subsection{Finding the solution}
\label{sect:finding_the_solution}

SHORT SECTION

Multilinear monomial detection has only been given as a detector for a solution,
i.e., a decision algorithm. 
\amnote*{Yes, but probably restrict it to $\sim1$ page (there is enough content
as it is)}{ Talk about actually finding the solution.  }

\cite{Koutis08} gives an algorithm that solves the decision problem for $k$-path. 
$k$-path is found with \bigOstar{n+min(k^2, m)} applications of the algorithm.

\cite{Williams09} solves the decision problem for $k$-path. \cite{Williams09} also gives an algorithm 
that finds a path when it is known that a $k$-path exists.

\clearpage

\section{Improving algebraic fingerprinting}

As mentioned in Section \ref{sect:limits}, the algebraic fingerprinting framework proposed by 
\citeauthor{Koutis08} and \citeauthor{Williams09} \cite{Williams09, KouWil15} for 
$k$-multilinear monomial detection has
\amnote*{$O$ and \enquote{lower bound} don't go well together}{a lower bound of \bigOstar{2^k}. }
However, this framework \amnote[inline,nomargin]{is very generic since it}
approaches the abstract multilinear monomial detection without 
utilizing the combinatorial properties specific to the underlying problem. 
The idea of adding auxiliary fingerprint variables in the algebrization, though, 
has potential for these properties. TODO: rephrase more clearly

Indeed,
\amnote*{for...?}{faster algorithms}
 have been found by designing new algebrizations with techniques 
similar to algebraic fingerprinting, where the fingerprints are designed to 
\amnote*{make use of}{abuse}
the underlying combinatorial properties. In \cref{sect:cancel_nonsolutions}, 
we show how \citeauthor{Björklund14} \cite{Björklund14}
exploited the cancelation due to characteristic to cancel non-solution monomials 
by clever design of fingerprints. 

Furthermore, the evaluation of the polynomial in the algebraic fingerprinting framework is done sequentially. 
However, the matrix representations of the group algebras used in \cite{Williams09} 
offer possibilities for parallelization.
In \cref{sect:parallelization}, \amnote[inline,nomargin]{we discuss} the ideas of 
\citeauthor{Midas19} behind the distributed multilinear monomial detection
\cite{Midas19} \amnote*[inline,nomargin]{}{(are discussed)}.

\subsection{Fingerprinting for cancelation of non-solutions}
\label{sect:cancel_nonsolutions}

TODO: go over Björklund et al. for k-path or Hamiltonicity, managed to design fingerprints such that non-solutions cancel

In the algebraic framework by \citeauthor{Koutis08} and \citeauthor{Williams09}, 
fingerprints prevent the cancelation of multilinear monomials. 
Attacking the Hamiltonian path problem, however, \citeauthor{Björklund14} 
designed fingerprints such that non-multilinear monomials, i.e. non-solution terms, 
cancel due to characteristic, while the multilinear terms 
remain with constant probability \cite{Björklund14}. This resulted in the current 
fastest algorithm for undirected Hamiltonicity, running in \bigOstar{2^n} time.

Before discussing the algebrization and the fingerprints, 
we define the Hamiltonian path problem.

\begin{problem}
  \problemtitle{\textsc{Hamiltonian path (Hamiltonicity)}}
  \probleminput{A directed graph $G = (V,E)$.}
  \problemquestion{Does $G$ contain a simple path that visits every vertex?}
\end{problem}

TODO: give the algebrization and/or show how it works

\subsection{Parallelizing multilinear monomial detection}
\label{sect:parallelization}

\amnote[inline,nomargin]{Probably very ambitious to describe the full thing
here... (You already need to focus on getting the details right in Sects. 4 and
5.1.) Maybe instead just give a broad picture of what other things people have
considered? (unless this is really the only other thing out there)}

TODO
\clearpage
\section{Conclusion}

TODO: conclude
%\clearpage                     % luku loppuu, loput kelluvat tänne, sivunv.

%\input{luku2}                  % tässä tyylissä ei sivunvaihtoja lukujen
%\input{luku3}                  %   välillä. Toiset ohjaajat haluavat 
%\input{luku4}                  %   sivunvaihdot.

\label{pages:text}
\clearpage                     % luku loppuu, loput kelluvat tänne, sivunvaihto
%\newpage                       % ellei ylempi tehoa, pakota lähdeluettelo 
                               % alkamaan uudelta sivulta

% -------------- Lähdeluettelo / reference list -----------------------
%
% Lähdeluettelo alkaa aina omalta sivultaan; pakota lähteet alkamaan
% joko \clearpage tai \newpage
%
%
% Muista, että saat kirjallisuusluettelon vasta
%  kun olet kääntänyt ja kaulinnut "latex, bibtex, latex, latex"
%  (ellet käytä Makefilea ja "make")

% Viitetyylitiedosto aaltosci_t.bst; muokattu HY:n tktl-tyylistä.
%\bibliographystyle{aaltosci_t}
% Katso myös tämän tiedoston yläosan "preamble" ja siellä \bibpunct.

% Muutetaan otsikko "Kirjallisuutta" -> "Lähteet"
\renewcommand{\refname}{\REFERENCES}  % article-tyyppisen
%\renewcommand{\bibname}{Lähteet}  % jos olisi book, report-tyyppinen

% Lisätään sisällysluetteloon
\addcontentsline{toc}{section}{\refname}  % article
%\addcontentsline{toc}{chapter}{\bibname}  % book, report

% Määritä kaikki bib-tiedostot
%\bibliography{sources}
%\bibliography{thesis_sources,ietf_sources}
\amnote{No need for URL if you have DOIs, also no need for ISSNs\\\mbox{}}
\amnote{You should use math style in the title of \cite{Williams09}}
\amnote{use dblp.org}
\amnote{added \texttt{maxbibnames} (ensures all authors get mentioned)}
\printbibliography

\label{pages:refs}
\clearpage         % erotetaan mahd. liitteet alkamaan uudelta sivulta

% -------------- Liitteet / Appendices --------------------------------
%
% Liitteitä ei yleensä tarvita. Kommentoi tällöin seuraavat
% rivit.

% Tiivistelmässä joskus matemaattisen kaavan tarkempi johtaminen, 
% haastattelurunko, kyselypohja, ylimääräisiä kuvia, lyhyitä 
% ohjelmakoodeja tai datatiedostoja.

%\appendix
%\section{Esimerkkiliite}
\label{sec:app1}

Jos työhön kuuluu suurikokoisia (yli puoli sivua) kuvia, taulukoita
tai karttoja tms., jotka eivät kokonsa puolesta sovi tekstin joukkoon,
ne laitetaan liitteisiin. Liitteet numeroidaan. Jokaiseen liitteeseen
tulee viitata tekstissä, eikä liitteisiin ole tarkoitus laittaa ``mitä
tahansa'', vaan vain työlle oikeasti tarpeellista
materiaalia. Liitteisiin voidaan sijoittaa esim. malli
kyselylomakkeesta, jolla tutkimushaastattelu toteutettiin,
pohjapiirustuksia, taulukoita, kaavioita, kuvia tms.

\textbf{TIK.kand suositus: Vältä liitteitä.} Jos iso kuva, mieti onko
sen koko pienettävissä (täytyy olla tulkittavissa) normaalin tekstin
yhteyteen. Joskus liitteeksi lisätään matemaattisen kaavan tarkempi
johtaminen, haastattelurunko, kyselypohja, ylimääräisiä kuvia, lyhyitä
ohjelmakoodeja tai datatiedostoja.

Työtä varten mahdollisesti tehtyjä ohjelmakoodeja ei tyypillisesti
lisätä tänne, ellei siihen ole joku erityinen syy. (Kukaan ei ala
kirjoittaa tai tarkistamaan koko koodia paperilta vaan pyytää sitä
sinulta, jos on kiinnostunut.)

%\subsection{Esimerkkiliitteen otsikko 1}
%\label{sec:app1_1}
%
%Kerätty data-aineisto.
%
% -------------------------------------------------------------- %
%
%\newpage
%\section{Toinen esimerkkiliite}
%\label{sec:app2}
%
%Haastattelukysymykset: mitä, missä, milloin, kuka, miten.



%\label{pages:appendices}

% ---------------------------------------------------------------------

\end{document}
