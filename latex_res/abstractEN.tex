% Tiivistelmät tehdään viimeiseksi. 
%
% Tiivistelmä kirjoitetaan käytetyllä kielellä (JOKO suomi TAI ruotsi)
% ja HALUTESSASI myös samansisältöisenä englanniksi.
%
% Avainsanojen lista pitää merkitä main.tex-tiedoston kohtaan \KEYWORDS.

\amnote{prefer \texttt{\textbackslash cdot}, $\ast$ usually indicates some other
operation \\[1em]} 
\amnote{use $\poly$ from complexity package}
\amnote{First sentence of abstract uses a bit too much jargon}
\amnote{two \enquote{howevers} in third paragraph}

\begin{enabstract}
  This thesis studies how parameterized combinatorial problems can be 
  solved by detecting a $k$-multilinear monomial in a multivariate polynomial 
  by algebraic means. More specifically, the thesis focuses on the technique 
  of algebraic fingerprinting by Koutis and Williams. Moreover, the thesis 
  overviews the technique as a general framework for parameterized combinatorial 
  problems, and discusses its limits. Finally, the thesis collects some ideas 
  for improving the framework or even surpassing the lower limit of the 
  general framework.

  With algebraic fingerprints, a $k$-multilinear monomial can be detected 
  in time \bigO{2^k * \poly(n)}, where $n$ is the number of variables in the 
  multivariate polynomial. This gives a general framework for solving parameterized 
  combinatorial problems efficiently, since they can in general be reduced to
  instances of $k$-multilinear 
  monomial detection problems. 
  Especially algorithms that rely on color coding can be accelerated in time 
  by a factor of $e^k$ with the use of algebraic fingerprints. Currently, 
  this technique underlies fastest algorithms 
  for many parameterized combinatorial problems. 

  However, \bigOmega{2^k * poly(n)} is the lower limit of the technique: 
  multilinear monomials cannot be detected faster 
  with the general technique of algebraic fingerprints by Koutis and Williams. 
  However, the algebraic framework solves the very general multilinear monomial 
  detection problem; if the underlying combinatorics of the problem are well understood 
  and taken use of in the reduction to multilinear monomial detection, 
  a much faster algorithm based on algebraic fingerprinting may be 
  found for a specific problem. Using ideas from algebraic fingerprinting, 
  Björklund managed to find an algorithm that 
  finds a Hamiltonian path in time \bigO{1.657^n * poly(n)}.
%
%Tiivistelmätekstiä tähän (\languagename). Huomaa, että tiivistelmä tehdään %vasta kun koko työ on muuten kirjoitettu.
\end{enabstract}
