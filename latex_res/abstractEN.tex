% Tiivistelmät tehdään viimeiseksi. 
%
% Tiivistelmä kirjoitetaan käytetyllä kielellä (JOKO suomi TAI ruotsi)
% ja HALUTESSASI myös samansisältöisenä englanniksi.
%
% Avainsanojen lista pitää merkitä main.tex-tiedoston kohtaan \KEYWORDS.
\begin{enabstract}
  This thesis studies how parameterized combinatorial problems can be 
  solved %by detecting a $k$-multilinear monomial in a multivariate polynomial 
  by algebraic means. More specifically, the thesis focuses on the technique 
  of algebraic fingerprinting by Koutis and Williams. The thesis 
  overviews the algebra behind the technique and discusses its limits 
  as a general framework for parameterized problems. 
  Finally, the thesis collects some ideas 
  for improving the framework or even surpassing its lower runtime boundary.

  Combinatorial problems can be reduced to instances of an algebraic problem 
  of detecting a multilinear monomial in a multivariate polynomial. 
  With algebraic fingerprints, a multilinear monomial can be detected 
  in time \bigO{2^k \cdot \poly(n)}, where $n$ is the number of variables in the 
  multivariate $k$-degree polynomial. This gives a general framework for solving parameterized 
  combinatorial problems efficiently. 
  Currently, this technique underlies the fastest algorithms 
  for many parameterized combinatorial problems. 

  Moreover, \bigOmega{2^k \cdot \poly(n)} is the lower limit of the technique: 
  multilinear monomials cannot be detected faster 
  with the general framework of algebraic fingerprints. 
  However, this framework solves the very general multilinear monomial 
  detection problem; if the underlying combinatorics of the problem are well understood 
  and taken use of in the reduction to multilinear monomial detection, 
  a much faster algorithm may be 
  found for a specific problem. Indeed, e.g. 
  Björklund found an algorithm that 
  finds a Hamiltonian path in time \bigO{1.657^n \cdot \poly(n)} by 
  using techniques from algebraic fingerprinting. Furthermore, 
  it seems that understanding and making use of the ideas behind algebraic fingerprints 
  is the key to improving from the general framework.
%
%Tiivistelmätekstiä tähän (\languagename). Huomaa, että tiivistelmä tehdään %vasta kun koko työ on muuten kirjoitettu.
\end{enabstract}
