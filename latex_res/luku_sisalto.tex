% --------------------------------------------------------------------

\section{Introduction}

In recent years, there have been rapid advances in the algorithms for combinatorial problems. 
This has been greatly sparked by the development in algebraic methods for solving the 
multilinear monomial detection problem, i.e., finding whether a multivariate polynomial contains a multilinear monomial. 
Namely, the technique of algebraic fingerprinting 
first introduced by Koutis in [ref] and further developed by Williams in [ref] has proven 
to be very successful. With algebraic fingerprinting, the k-path problem (see problem def), 
that previously could be solved in \bigO{something} time by X in [ref], 
could be solved in \bigO{2^k*poly(n)} time by Koutis in [ref].\nl

Of course, this method was further developed, and in [ref] Björklund et al. showed an algorithm 
that solved the Hamiltonian problem (Hamiltonicity), i.e., finding whether a given graph contains a path that visits 
every vertex once, in \bigO{something} time. The previous fastest algorithm for Hamiltonicity by Y 
in [ref] ran in \bigO{something} time with the use of color coding[??]. [INSERT PREVIOUS MTEHOD] 
This was a significant improvement on a problem that had seen no progress in nearly forty years.\nl

The method of algebraic fingerprinting is present in multilinear monomial detection. Multilinear monomial detection 
is a fundamental problem, since many important combinatorial problems can be reduced into multilinear monomial detection 
via a problem specific algebraization. The goal of such algebraization is to form a generating multivariate polynomial 
that encodes the combinations, i.e. the solutions and non-solutions, into multivariate monomials where 
multilinear monomials correspond to solutions to the given problem.\nl

Before discussing multilinear monomial detection and algebraic fingerprints, the thesis covers algebraization, and 
shows how a combinatorial problem can be reduced into a multilinear monomial detection problem. 

\subsection{Algebraization of combinatorial problems}

A combinatorial problem asks whether a given finite set of objects satisfies some given constraints. 
For example, the k-path problem asks for, given a finite set of vertices and edges, 
a path of k vertices. The solutions and non-solutions to combinatorial problems can be thought of as 
combinations of the given objects. The solution space for the k-path problem consists of combinations of k vertices and k-1 edges. 
A non-solution combination would contain duplicate vertices or edges that contain vertices outside the combination.\nl

Algebraization is reducing a given problem into an algebraic problem, i.e., a question regarding some algebraic property of some algebraic entity. 
In an algebraization of a combinatorial problem, the algebraic entity can be constructed from algebraic elements defined from the 
set of objects given as an input. The motivation behind the construction is some algebraic property that, 
when satisfied, gives a solution to the problem.\nl

[TODO: rewrite this paragraph, explain with generating polynomial and expasion into sum of monomials]
Multilinear monomial detection has proven to be a useful algebraization. First, we introduce multiple variables that 
correspond to elements from the set of objects given as input. Then, we construct a multivariate polynomial such that it 
encodes all solutions and non-solutions as multivariate monomials, with solutions encoded specifically as multilinear monomials. 
Thus, the task of finding a satisfying combination to the combinatorial problem has been reduced to 
finding a multilinear monomial from the multivariate polynomial. It follows that a decision problem is be answered by 
the existence of a multilinear monomial.\nl

Appropriate definitions for the variables are problem specific. [ref] showcases an example 
reduction into multilinear monomial detection for the k-3D matching problem.\nl 

[TODO: define k-3d matching and go through the example]

The detection of multilinear monomials is a fundamental problem, since many important problems have been reduced to it. 
A faster algorithm for an instance of multilinear monomial detection is likely to inspire faster algorithms for 
other multilinear monomial detection problems [EXAMPLE HERE]. Of course, simply expanding the generating polynomial into 
a sum of monomials is not an optimal solution.[Are there any other methods that can be applied generally?]\nl

When the problem domain has n variables in an N-degree polynomial, the number of possible monomials is \(\binom{n+N}{n}\).
This motivates the detection of multilinear monomials without fully expanding the polynomial into a sum of monomials, 
which will be the topic of the next section.

\subsection{Johdantoluku}
% Sanni Heinzmann

Työn ensimmäinen luku on aina johdanto. Kandidaatintyön laajuudessa
sitä ei ole tarvetta jakaa alalukuihin, diplomityössä ja muissa
isommissa töissä sekä tutkimusraporteissa alaluvut ovat mahdollisia
(esim. 1.1 Tutkimusongelma, 1.2 Aineisto ja tutkimusmenetelmä, jne.).

Johdannon tarkoitus on antaa lukijalle heti alussa selvä kuva siitä,
mihin kysymykseen työ pyrkii vastaamaan
(tutkimusongelma). Tyypillisesti aiheen esittely alkaa sanoilla ``Tämä
(kandidaatin/diplomi)työ käsittelee...''  Johdanto esittelee lyhyesti
työn pääpiirteet ja johdattaa lukijan itse työn pariin. Johdannon
ohjepituus on 1--3 sivua, kandidaatintyössä 2 sivua on hyvä maksimi.

Käsittele nämä aiheet johdannossa (jotakuinkin tässä järjestyksessä):
%
\begin{itemize}
 \item Johdatus aihepiiriin 
(ei liian laajasti, vaan relevantisti ja napakasti)
%
 \item Tutkimuskohteen esittely (MITÄ tämä työ tutkii? 
Kerro työstä/tutkimusaiheesta, ei omasta kirjoitusprosessistasi 
tai omasta kiinnostuksestasi.)
%
 \item Tutkimuksen perustelu: ongelma tai aukko 
(aiemmassa tutkimuksessa on aukko, tai siitä nousee esiin 
kysymys, johon tässä etsitään vastausta)
%
 \item Tutkimusongelma / -kysymykset (koko työsi sydän, 
jonka pitäisi näkyä ''punaisena lankana'' koko työn läpi)
 \item Tavoitteet (Käytä konkreettisesti sanaa ''tavoite'')
 \item Rajaus (Mitä tämä työ EI tutki)
 \item Menetelmä, aineisto, teoreettinen kehys (Esittele, 
MITEN em. aihetta tutkitaan)
 \item Tulokset? (Johdannossa on ihan hyvä antaa lyhyesti 
tietoa päätuloksista, mutta ei pakko)
 \item Työn sisältö ja rakenne (Esittele, miten työn punainen 
lanka etenee, viittaukset työhön: 
``ensin, sitten, seuraavaksi, luvussa 3'' jne.)
\end{itemize}

%\textbf{TIK.kand kommentti: Helpota lukijan työtä kaikin
%mahdollisin tavoin. Haluat voittaa lukijan omalle puolellesi.
%Lukijalle on hyvä tuoda heti alusta selville, mihin pyrit.
%Verratuna kaunokirjallisuuteen: ``kerro heti kuka murhaaja on''}

% --------------------------------------------------------------------

\section{Kandidaatintyön rakenne- ja muotoseikat}
\label{sec:esimluku}

Tässä luvussa esitellään kandidaatintyön muotovaatimuksia
tällä kurssilla. Muutamat alkuperäiset lähteet ovat saattaneet
kadota organisaatio- ja tietojärjestelmämuutoksissa, kun
Into-järjestelmä on korvannut WWW-sivustoja.

\subsection{TKK:n kandidaattityöryhmän ohjeistus}

Yleiset kandidaatintyön muotovaatimukset on annettu TKK:n
kandidaattityöryhmän päätöksellä 14.11.2006 ja ne ovat
kokonaisuudessaan saatavissa osoitteessa
\url{http://www.tkk.fi/fi/opinnot/opintohallinto/paatokset/kandi20061114.pdf}.
Tässä luvussa annetaan lyhyt, selvennetty ja joiltakin osiltaan
karsittu yhteenveto kyseisistä ohjeista. Seuraavassa viitataan siis
edellä mainittuihin TKK:n kandidaatintyön ohjeisiin (esim. ``luku 3''
tarkoittaa TKK:n kandidaatintyön ohjeiden lukua kolme).

\textbf{TIK.kand suositus: Lue alkuperäiset ohjeet erityisesti
  silloin, jos et kirjoita työtä annettua \LaTeX{}-pohjaa käyttäen.}

TKK:n kandidaatintyön ohjeissa käsitellään työn rakennetta (luvussa 3)
ja muotoseikkoja (luvussa 4). Yleisesti todetaan kandidaatintyöstä
seuraavaa:
%
\begin{quotation}
\noindent \it
Kandidaatintyö voi perustua teoreettisen taustan tarkasteluun 
ja sen analysointiin sekä johtopäätösten tekoon tai kokeelliseen osioon ja 
tulosten analysointiin sekä johtopäätösten tekoon
tai edellisten yhdistelmään.
Kandidaatintyön rakenteen tulee olla hyvän tieteellisen kirjoittamisen 
käytännön mukainen
ja sisältävän vähintään seuraavat osat: [$\ldots$] (Luku 3)
\end{quotation}

Rakenteen osia ovat: nimiölehti, tiivistelmä, sisällysluettelo,
symboli- ja lyhenneluettelo (työn luonteen vaatiessa voi puuttua),
johdanto, aikaisempi tutkimus (työn luonteen vaatiessa teoreettinen
tausta), tutkimusongelma ja -menetelmät, tulokset, tarkastelu (työn
luonteen vaatiessa johtopäätökset tai näiden yhdistelmä), lähteet,
liitteet (jos tarpeen). Osat johdannosta tarkasteluun muodostavan työn
tekstiosan. (Luku 3) Tekstiosan sopiva pituus on 15--20 sivua eikä
työtä ole syytä tarpeettomasti pidentää (luku 4.2.1).
Kokonaissivumäärä tulee tällöin olemaan noin 18--25 sivua.

Muotoseikoissa TKK:n kandidaatintyön ohjeissa otetaan esille, että
työn tulee olla jäsennelty ja tyylillisesti sekä kielellisesti
viimeistelty ja moitteeton.  Tarpeettomia tyylillisiä erikoisuuksia
tulee välttää.  Tekstiosassa tulee olla vain työn kannalta oleelliset
kuvat tai taulukot. (Luku 4.1)
%
Kirjasinlaji tulee olla roomalaistyyppinen (Times New Roman
tai Computer Modern\footnote{\LaTeX{}in perusfontti}) ja kooltaan 12 pistettä
(luku 4.2.2). 
%
Työn nimessä ei saa esiintyä lyhenteitä, kaavoja tai lainauksia
(luku 4.2.3).

Nimiölehdellä tulee olla tiedot yliopistosta, tutkinto-ohjelmasta,
työn nimestä, luonteesta (``Kandidaatintyö''), päivämäärä ja tekijän
nimi (luku 4.2.4). Tiivistelmäsivusta esitetään vastaavat seikat,
jotka löytyvät myös tarjolla olevista pohjista
(doc\footnote{\url{http://peppi.hut.fi/pub/kandi/kandi.php}} tai tämä
\LaTeX{}-pohja).  Se ei saa olla sivua pidempi. 
% WANHA:
%Tiivistelmässä
%ilmoitettavaan sivumäärään lasketaan tekstiosan lisäksi
%lähdeluetteloon kuluvat sivut. (Luku 4.2.5)
Tiivistelmässä
ilmoitettavaan sivumäärään lasketaan kaikki sivut yhteen 
nimiölehdestä lähdeluettelon tai liitteiden loppuun
(Kirjaston suullinen ohje 29.8.2011). 

Asemoinnin suhteen tekstiosaa ei sisennetä vaan kappaleiden väliin
jätetään yksi tyhjä rivi. Jos oikea reuna tasataan, niin tulee käyttää
tavutusta ja tarkistaa, että se menee oikein.  Rivivälin tulee olla 1
tai 1,5. (Luku 2.4.7)

Huomaa, että TKK:n kandidaatintyön ohjeissa sivujen numeroinnin ohjeet
ovat ristiriitaiset (luku 4.2.6). Oikea sivunumeroinnin malli on
toteutettu tässä pohjassa.  

Lähdeviittaukset tulee tehdä huolellisesti ja samanmuotoisesti joko
nimi-vuosi- tai numerojärjestelmällä. Alaviitejärjestelmää ei
suositella. (Luku 2.4.8)

\textbf{TIK.kand suositus: Numerointi aloitetaan arabialaisilla
  numeroilla nimiölehdestä kuitenkin niin, ettei numeroa kirjoiteta
  sille. Siten ensimmäisen tiivistelmäsivun sivunumero on 2. 
  Kirjasinkoko on 12, riviväli 1,5. Tekstiviitteissä käytetään
  nimi-vuosi-järjestelmää, mutta tässä ohjaajan sana on määräävä.}

\subsection{TIK.kand: kommentteja rakenne- ja muotoseikoista}

Tekstiasun viimeistelyyn tulee varata runsaasti aikaa V3- ja
V4-palautusten väliin. Työn tulisi olla oleellisesti valmis jo
V3-palautuksessa, jotta ohjaajasi voi antaa palautetta, kuinka työ
viimeistellään ja saadaan hyväksi kokonaisuudeksi, ja sinulla on 
tarpeeksi aikaa hienosäätöön.

Huomaa, että sivumäärä on hyvin riippuvainen tekstin luonteesta:
pelkkää tekstiä mahtuu tässä \LaTeX{}-pohjassa sivulle noin 300 sanaa,
kun taas jos mukana on kuvia, luetteloita tai paljon väliotsikkoja,
sanamäärä on huomattavasti pienempi. Vastaavasti pienempi riviväli tai
fonttikoko antaa lisää sanoja sivua kohden.

Luennolla 6.9.2011 kurssin vastuuopettaja Tomi Janhunen linjasi, että
sivumäärä ei ole kriittisin asia vaan itse aiheen käsittely.  Jos
sivumäärä poikkeaa, ohjaaja tai kurssin henkilökunta voi puuttua
tilanteeseen. Jos sivumäärä on pieni, voidaan kysyä, onko aihetta
käsitelty tarpeeksi laajasti tai onko selittämiseen tai perusteluihin
käytetty vaivaa. Toisaalta iso sivumäärä voi kertoa siitä, ettei työn
rajaamista ole hallittu, ja tämäkin voi olla arviointiin liittyvä
tekijä. Myöskään ei saa ajatella sitä, että kun 20 sivua tekstiä tulee
täyteen, niin opinnäytetyö on valmis. Tieteellisen tekstin
kirjoittaminen on iteratiivista: kirjoitetaan, luetaan, karsitaan ja
lisätään, kirjoitetaan uudestaan, luetaan, jne. Tyypillisesti
sivumäärän kanssa ei tule ongelmaa: annetusta tehtävästä tällä
aikataululla tulee tyypillisesti noin 20 sivun mittainen raportti.

Ohjeita kandidaatintyön eri osien kirjoittamiseen on sisällytetty
tähän pohjaan. Kirjoittamisohjeita on koskien tiivistelmää, käytetyt
lyhenteet -osiota, johdantoa, loppulukua ja liitteitä.  Lisäksi
lähdekoodissa \verb!main.tex! on ohjeita alkusanoihin, joiden käyttöä
ei suositella kandidaatintyössä.

\subsection{Kirjallisuutta}

Seuraavat kolme kirjaa löytyvät T-kirjaston käsikirjastosta:
%
\begin{itemize}
\item \cite{kauranen2006} ovat kirjoittaneet kirjan erityisesti TKK:n
  opinnäytetöitä ajatellen.  Kirjaa saa pääkirjastosta 8 euron
  hintaan.
%
\item ``Tutki ja kirjoita'' \citep{hirsjarvi2009} lienee Suomessa alan
  perusteos.  Kirjan hinta on noin 60--70 euroa.  T-talon kirjastossa
  on kohdassa ``Yleistä 0'' muutamia vanhempia painoksia kirjasta.
%
\item Kielenhuollon opas \citep{oikeinkirjoitus2010} tarjoaa apua
  oikeinkirjoitukseen. Kirjaa saa Kotimaisten kielten
  tutkimuskeskuksen verkkokaupasta
  \url{http://www.kotus.fi/index.phtml?s=2420} 20 euron hintaan.
\end{itemize}

Kurssiesitteeessä Nopassa on myös linkkejä lukuisiin 
kotimaisiin Internet-sivustoihin opinnäytetyön kirjoittamisesta.
Näitä ovat mm.
%
\begin{itemize}
\item Oulun yliopiston ``Kirjoittamisen ABC''
  \url{http://webcgi.oulu.fi/oykk/abc}
%
\item Helsingin yliopioston puhe- ja kirjoitusviestinnän opas
  ``Kielijelppi'' \url{http://www.kielijelppi.fi/}
%
\item Oman kirjastomme tarjoama tiedonhaun itseopiskelupaketti
  \url{http://peppi.hut.fi/pub/opetus/tiedonhaku/pmwiki.php}
%
\item Yucca Korpela on kirjoittanut nettioppaan ``Arkisen
  asiakirjoittamisen opas'' \url{http://www.cs.tut.fi/~jkorpela/kirj/}
\item Aalto-yliopiston Opiskelutaidot-sivusto \\
  \url{https://into.aalto.fi/ display/ fiopiskelutaidot/}
%
\end{itemize}


% --------------------------------------------------------------------

\section{Esimerkkejä \LaTeX{}in käytöstä}
\label{sec:esimluku}

Tässä luvussa annetaan esimerkkejä tyypillisimpiin
kirjoitustehtäviin. Katso siis valmista PDF-tiedostoa ja lähdetekstiä
tiedostossa \verb!luku_sisalto.tex!. Katso myös varsinaista
päätiedostoa \verb!main.tex! ja etenkin sen alkua, jossa ladataan
lisäpaketteja (sisältäen komentoja). Tämän dokumentin sivuasettelut
tehdään pääosin tyylitiedostossa \verb!aaltosci_t.sty!, jota ei tulisi
itse muuttaa lainkaan.

Tarkempia ohjeita voi etsiä kirjallisuudesta tai Internetistä
sopivilla hakusanoilla. Apua suomeksi: \citet{lyhyt2e},
\citet{mattakivela}, Wikibooksin LaTeX-opas%
\footnote{http://en.wikibooks.org/wiki/LaTeX/}, Jukka Korpelan
LaTeX-sivut%
\footnote{http://www.cs.tut.fi/~jkorpela/softa/latex.html} yms.
Järkäleteoksia, mm.  \citet{mittelbach2004}, on saatavilla myös
kirjaston sivujen kautta e-kirjana. Googlaamalla ``latex <ongelmasi
avainsanoja>'' löytyy varmasti apua.

\subsection{\LaTeX{}in asennus ja taustaa}
\label{sec:esimlatexajo}

TeX-jakeluita on saatavilla ``kaikkiin'' eri ympäristöihin.
Suositeltavaa (helpointa?) on käyttää koulun omia Linux-ympäristöjä,
jolloin tarvittavat tausta-asetukset lienevät kunnossa.  Windows- ja
Mac-koneille on saatavana eri TeX-jakeluja, mm.
TeXlipse\footnote{\url{http://texlipse.sourceforge.net/}} (Eclipsen
liitännäinen) ja MiKTeX\footnote{\url{http://miktex.org/}}.

\subsubsection{Lähdetiedostosta PDF:ksi}
\label{sec:esimkaannos}

Tässä zip-paketissa on mukana \verb!Makefile! (päivitä
omat tex- ja bib-tiedostojen nimet), joten pelkkä komento
\verb!make! riittää. Olkoon tässä päätiedoston nimi \verb!main.tex! --
voit sen vaihtaa luonnollisesti miksi tahansa.

Jos ajat \LaTeX{}ia komentoriviltä tai jostain graafisesta ikkunasta,
niin ``käännä ja kaulitse'' \verb!pdflatex main.tex! tarvittaessa kaksikin
kertaa. Kun olet lisännyt tekstiviitteitä komenna \verb!pdflatex main!,
\verb!bibtex main!, \verb!pdflatex main!, \verb!pdflatex main!. Tarkkaile
ruudulle tulevaa tulostusta; esimerkiksi:
%
\begin{verbatim}
Package natbib Warning: Citation(s) may have changed.
(natbib)                Rerun to get citations correct.
\end{verbatim}

Käännösvaiheissa hakemistoon ilmestyy monenlaisia työ- ja 
lokitiedostoja, joiden päätteinä mm. aux, log, toc, bbl, blg.
Joskus voi olla syytä poistaa nämä komentamalla \verb!make clean!.

\subsubsection{Ongelmien ratkaisija: \LaTeX{} checker}
\label{sec:esimlacheck}

Sopiva tekstieditori (emacs, TeXLive, LEd, $\ldots$) osaa
neuvoja, kun joku tekstissä on joku kielioppivirhe. Tämän 
lisäksi oiva työkalu on \verb!lacheck!, joka löytyy (?)
unix-koneista asennettuna ja ladattavissa 
Windows-koneelle%
\footnote{\url{http://www.ctan.org/tex-archive/support/lacheck/}}.
Komennon \verb!lacheck main.tex! (\verb!lacheckw32 main.tex!)
tulostuksesta voi helposti etsiä, missä kohtaa on jäänyt joku
sulku tai ympäristö sulkematta kiinni. Alla olevassa listauksessa
vika löytyy rivin 224 läheisyydestä.
%
\small
\begin{verbatim}
** luku_sisalto:
"luku_sisalto.tex", line 178: missing `\ ' after "engl."
"luku_sisalto.tex", line 224: <- unmatched "\end{center}"
"luku_sisalto.tex", line 1: -> unmatched "beginning of file luku_sisalto.tex"
"luku_sisalto.tex", line 506: <- unmatched "end of file luku_sisalto.tex"
"main.tex", line 48: -> unmatched "\begin{document}"
\end{verbatim}
\normalsize

\subsubsection{``Ääkköset eivät ole enää ongelma''}

Katso tiedoston \verb!main.tex! alkua. Näppäimistön merkistökoodaus
valitaan kohdassa \verb!inputenc!. Kaikkien lähdetiedostojen tulee
olla saman merkistökoodauksen mukaisia.  Useat editorit osaavat
vaihtaa koodausta; pääosin on tarve vaihtaa ISO-8859-1 (Latin 1) ja
UTF-8 (Unicode) välillä. Linuxissa voit katsoa tiedoston koodauksen
\verb!file -i tiedosto.tex! ja muuttaa sen tarvittaessa\\
\verb!iconv -f ISO_8859-1 -t UTF8 fileLatin1.tex > fileUTF8.tex!.

Kohdassa \verb!fontenc! kerrotaan, millaista ulostuloa \verb!pdflatex!in
halutaan antavan. Tämä näkyy esimerkiksi siinä, ovatko kirjaimet
bittikarttoja vai vektorigrafiikkaa (suurenna PDF-selaimessa 1600\%)
tai miten ääkköset esitetään (kopioi ja liitä tekstiä ruudulta 
tekstieditoriin; näkyykö ä \verb!ä!:nä vai \verb!\"a!:na.

Tämän zip-paketin tiedostot ovat UTF8-koodattuja.

\subsubsection{Tavutus ei toimi?}
\label{sec:esim_tavutus}

\LaTeX{} osaa tavuttaa melko lailla oikein, kun valitaan \verb!babel!illa
oikea kieli. Joidenkin hankalien sanojen osalta voit auttaa
ehdottamalla tavurajoja paikallisesti \verb!ta\-vu\-ra\-ja!
tai koko tekstin osalta \verb!\hyphenation{}!-määrittelyssä
\verb!main.tex!:n alussa. Jos tavutus ei toimi, varmista merkistökoodaus
(UTF-8 / ISO-8859-1). Varmista myös, että valittuna tekstissä oikea kieli
komennolla \verb!\selectlanguage!.

Testaa myös kääntöä IT-keskuksen koneissa -- jos
toimii koululla, niin omasta jakelusta puuttuu \verb!babel!.
Katso myös luvun~\ref{sec:hienos} \verb!sloppypar!-ympäristö.

\subsubsection{Oikoluku}

Oikoluvun suoran tuen puute on yksi iso ongelma kirjoitettaessa suomeksi.
Yksi mahdollisuus on kopioida teksti johonkin oikolukijaan. Helpompi
tapa lienee kopioida tiedostot Linux-koneille, joissa
suomenkielisen tekstin voi oikolukea Voikkoa käyttäen tex-tiedostoista
\verb!tmispell -dsuomi -t main.tex!.
Ohjelman \verb!tmispell! vipu \verb!-t! jättää \LaTeX{}-komennot
huomioimatta. Ohjelma saattaa lukea vain UTF8:aa, joten 
tällöin tiedostot on muutettava tai kopioitava \verb!iconv!illa,
katso ääkköslukua yllä.

\subsubsection{Hienosäätö}
\label{sec:hienos}

\begin{sloppypar}
  Vihoviimeisen version osalta tulee tarkastaa mm. tavutus
  (luku~\ref{sec:esim_tavutus}) ja rivien siisti ulkoasu.  Jos rivillä
  on kaavoja tai eri fontteja, rivi saattaa jatkua pitkäksi. Tällöin
  yksi mahdollisuus on käyttää \verb!sloppypar!-ympäristöä, joka antaa
  \LaTeX{}ille lisää vapautta päättää sanojen väleistä (katso
  lähdekoodi). Komento \verb!\sloppy! antaa väljyyden koko tekstiin.
  \verb!\hyphenpenalty!-arvon määrittämisen pitäisi myös
  auttaa (?).
\end{sloppypar}

Jos luvun $N$ kuvat tai taulukot ``valuvat'' lukuun $N+1$,
voi luvun loppuun kokeilla \verb!\clearpage! tai 
\verb!\afterpage{\clearpage}!, minkä tarkoituksena pakottaa
kelluvat objektit tulostumaan ennen luvun loppua.


\subsubsection{Eräs vaihtoehto Win7-koneella: MiKTeX ja LEd, Notepad++}
\label{sec:esimmiktex}

Esimerkin omaisesti esittelen oman kokonaisuuteni, johon kuuluu
Windows 7 -koneella MiKTeX ja
LEd-editori\footnote{\url{http://www.latexeditor.org/}}.  Jos kaikkia
paketteja (engl. package) ei ole ladattu valmiiksi, niin ne kannattaa
hakea erikseen MiKTeX package managerilla, joka löytyy nimellä
\verb!mpm!.

Ongelmia ja ratkaisuja: Jos paketti on puuttunut ja LEdin kautta
lataus epäonnistuu, niin avaa \verb!mpm! ja lataa sitä kautta.  Jos
tavutus ei toimi, niin tarkista, että sopiva
\verb!miktex-hyphen!-paketti on ladattu ja lisäksi suomen kieli on
valittu MiKTeXin konfiguraatiossa.

Usein kirjoitan tekstiä Notepad++-editorilla%
\footnote{http://notepad-plus-plus.org/}, joka on avoimen lähdekoodin
ohjelma. Sen jälkeen kopioin tiedoston koulun koneelle, jossa ajan
komennon \verb!pdflatex! tai \verb!make!.

\subsection{Tekstin kirjoittaminen}
\label{sec:esimmuotoilut}

Voit kirjoittaa tekstisi suoraan \verb!main.tex!-tiedostoon tai
vaikkapa luvuittain omiin tiedostoihin, jotka voi upottaa
päätiedostoon \verb!\input!-komennolla. Tässä käytetään
dokumenttityyppiä \verb!article!, joka on monessa suhteessa kevyempi
ja sopivampi kuin \verb!report! tai \verb!book!.


\subsubsection{Perusteksti ja muotoilut}

% Kokeile mitä kappaleen viimeisen rivin lopulle käy, 
% kun sloppypar-ympäristön kommentoi pois.
\begin{sloppypar}
Perusteksti kirjoitetaan konstailematta samalla fontilla ja koolla
läpi dokumentin. Sanojen \textbf{vahvistamista} tai
\textit{kursivointia} tulee välttää.  Riviväli on oletusarvoisesti
1,5, mutta sen voi halutessaan vaihtaa väliksi 1 kommentoimalla
\verb!main.tex!in alussa \verb!linespread!-rivin.  Fontin koko, 12 pt,
on asetettu heti \verb!main.tex!:n alussa
\verb!\documentclass!-määrittelyssä.
\end{sloppypar}

% tik.kand 1.5 ja 12

Lainatun tekstin tulee erottua selkeästi omasta tekstistä. Lainauksia
tulee käyttää maltillisesti. Asiat tulisi kertoa aina omin
sanoin. Lainaus merkitään tyypillisesti tekstin seassa
lainausmerkkeihin, jotka kirjoitetaan tyypillisesti kahdella
erillisellä merkillä \verb!''! tavallisten lainausmerkkien \verb!"!
sijaan. (Ainakin) emacs osaa muuttaa automaattisesti lainausmerkin
oikeanlaiseksi. Isompi lainaus voidaan sijoittaa
\verb!quotation!-ympäristöön:

\begin{quotation} { 
\noindent \it
IP Datacasting is a service where digital content formats, software
applications, programming interfaces and multimedia services are
combined through IP (Internet Protocol) with digital
broadcasting \citep{ipdcforum_def}. } 
\end{quotation}

Prosenttimerkkiä (\%) eikä mitään sen oikealla puolella olevaa tulostu
ulostuloon. Sillä voi katkaista pitkän rivin ja jat% katkaistaan rivi
kaa seuraavalta (katso lähdetiedostoa!).  Yhdysmerkki eli yhdysviiva
(-) saadaan yhdellä, ajatusviiva (--) kahdella peräkkäisellä
yhdysmerkillä: 7--9-vuotiaat, LaTeXin käyttö -- helppoa vai hullun
hommaa, 25-vuotias.

\subsubsection{Luetelmat}
\label{sec:esimluettelo}

Älä käytä luetelmia jatkuvasti kandidaatintyössäsi. Kirjoita
mieluummin suoraa tekstiä.

\LaTeX{}in peruslistaympäristöt ovat \verb!itemize!, 
\verb!enumerate! (numeroitu)
ja \verb!description!. Listoja ja niiden ulkoasua on
mahdollista muokata~\citep[katso esim.][s. 128]{mittelbach2004}. 
Perusesimerkkejä:
%
\begin{itemize}
\item luetteloaihe yksi
 \begin{itemize}
 \item luetteloaiheen sisäinen lista
 \end{itemize}
\item luetteloaihe kaksi
\end{itemize}

Toinen luettelo omine merkkeineen:
%
\begin{itemize}
\item[b)] luetteloaihe b
\item[E)] luetteloaihe E
\end{itemize}

Numerointi ympäristössä \verb!enumerate!:
%
\begin{enumerate}
\item luetteloaihe yksi
\item luetteloaihe kaksi
\end{enumerate}


Yksinkertainen taulukko, joka ei ole kelluva ja siten sen pitäisi
latoutua heti tämän tekstin alle. Tässä komento \verb!\topcaption!
kyllä varaa itselleen ``Taulukko 1'', mutta tekstiä ei näy missään.

\topcaption{Yksinkertainen taulukko}
%\begin{center}                         % keskitys?
\begin{tabular}{|l|l|l|} 
\hline
Tässä & sarakkeet & ei ole eroteltu mitenkään \\
ja    & ne        & saattavat valua yli laidankin ikävästi \\
\hline
\end{tabular}
%\end{center}

Toinen kelluva perustaulukko, viitataan nyt taulukkoon~\ref{table:perustaulu}.
Lähdetekstiä lukiessa näet tilde-merkin, joka pakottaa välilyönnin
mutta estää rivinvaihdon.

\begin{table}[ht]
\caption{Tässä perustaulukko.}
\label{table:perustaulu}
\begin{center}
\begin{tabular}{|l|l|l|} 
\hline
Tässä & sarakkeet & ei ole eroteltu mitenkään \\
\hline
ja    & ne        & saattavat valua yli laidankin ikävästi, %
                   siksi käytä taulukkoa~\ref{table:dvbt_param}. %
                   Teksti katoaa jonnekin sivun ulkopuolelle. \\
\hline
Nro   & Nro       & Nro \\
\hline
$-4$  & $8$       & $12$ \\
\hline
\end{tabular}
\end{center}
\end{table}

Vielä kolmas esimerkki taulukosta, jossa sarakkeiden leveys määritelty
ja soluissa voi olla useampi rivi tekstiä. Katso
taulukko~\ref{table:dvbt_param}.

\begin{table}[th]
\caption{The DVB-T transmission parameters.}
\label{table:dvbt_param}
\begin{center}
\begin{tabular}{|p{0.35\textwidth}|p{0.45\textwidth}|} 
    \hline
Parameter & Typical values \\
    \hline
    \hline
Physical channel&8 MHz (also 6 MHz or 7 MHz possible)\\ 
    \hline
COFDM mode (number of subcarriers, 
subcarrier width, 
signal element length)
&8k (6817, 1116 Hz, 896 $\mu$s) or 
2k (1705,4464 Hz, 224 $\mu$s)\\
    \hline
Guard interval (8k/4k duration)
&1/4 (224/56 $\mu$s), 1/8 (112/28 $\mu$s),
   1/16 (56/14 $\mu$s) or 1/32 (28/7 $\mu$s)\\
    \hline
Inner code rate &1/2, 2/3, 3/4, 5/6 or 7/8\\
    \hline
Signal  element constellation 
&QPSK, 16-QAM or 64-QAM \\
    \hline
\end{tabular}
\end{center}
\end{table}


% Tässä esimerkki monisivuisesta taulukosta.
% EI toiminut aivat täydellisesti 27.1.2011: tablecaption hyppäsi edelliselle
% sivulle ennen ylläolevaa taulukkoa. Tämä supertabular taas ei voi olla 
% table-ympäristössä, kun muuten ei sivutu. supertabular on vain tabular,
% joka pitää kirjaa milloin sivun vaihto ja lisää automaatigisesti \end{tabular}
% ja uudestaa \begin{tabular}.
%\begin{center}
%\tablecaption{Esimerkki monisivuisesta taulukosta.}
%%\topcaption{taulukon nimi ylhäällä}
%%\bottomcaption{taulukon nimi alhaalla}
%\tablefirsthead{\hline \# & Sarakkeen aihe \\ \hline \hline }
%\tablehead{\hline \multicolumn{2}{|l|}{jatkoa edelliseltä sivulta} \\ %
%  \hline  \# & Sarakkeen aihe  \\ \hline \hline }
%\tabletail{\hline \multicolumn{2}{|r|}{jatkuu seuraavalle sivulle} \\ \hline}
%\tablelasttail{\hline}
%\label{table:pitkataulukko}
%\begin{supertabular}{|>{\bf}r|p{120mm}|} 
%\hline
%\multicolumn{2}{|l|}{Otsikko A} \\ 
%\hline
%1 & complex numbers, Carthesian and polar coordinate systems, Euler's formula   %\\
%2 & Euler's formula, cosine and sine, odd and even functions   \\
%3 & complex numbers, graphical notation   \\
%\hline 
%\multicolumn{2}{|l|}{Otsikko B} \\ 
%\hline
%14 & analog, discrete-time and digital signal   \\
%17 & analog, discrete-time and digital signal   \\
%16 & digital signals and spectra, spectogram \\
%17 & Fourier series, Fourier transforms: CTFT, DTFT, DFT \\
%18 & time-frequency-domain analysis and filtering \\
%21 & moving average (MA) filter, a simple FIR filter   \\
%22 & a simple IIR filter \\
%23 & flow / block diagram of a discrete-time system   \\
%24 & recognition of LTI systems, causal LTI systems, filter order, FIR, IIR   %\\
%25 & properties of LTI systems: linear, time-invariant, causal, stable   \\
%26 & shifted and scaled sequences in LTI system   \\
%31 & convolution as products of polynomials   \\
%32 & deconvolution   \\
%33 & parallel and cascade (series) LTI systems   \\
%34 & matched filter   \\
%35 & auto- and cross-correlation   \\
%37 & spectrum, CTFT, discrete-time Fourier transform (DTFT), %
% discrete Fourier transform (DFT)   \\
%38 & DTFT, computation from definition   \\
%39 & DTFT, using a transform table   \\
%40 & $2\pi$-periodic spectrum, DTFT   \\
%41 & magnitude/amplitude response, periodicity of DTFT   \\
%43 & sampling, Shannon's theorem   \\
%44 & impulse train and Fourier-series   \\
%45 & impulse train and sampling in frequency-domain   \\
%46 & sampling in frequency-domain   \\
%47 & aliasing   \\
%48 & sampling, aliasing, anti-aliasing   \\
%49 & anti-aliasing filter  \\
%51 & circular shift, DFT \\
%54 & amplitude response grafically from pole-zero-plot   \\
%55 & analysis of LTI IIR system, pole-zero plot   \\
%56 & transfer function, region of convergence (ROC)   \\
%57 & scaling factor   \\
%63 & direct form (DF) structures   \\
%70 & FIR-window method in digital filter design   \\
%71 & computational comparisons between IIR and FIR filters   \\
%73 & FFT computational complexity   \\
%74 & radix-2 DIT FFT algorithm   \\
%75 & binary addition and substraction, two's complement  \\
%78 & roundoff noise in FIR filters   \\
%79 & signal-to-noise ration (SNR) \\
%80 & error-feedback structure   \\
%81 & up- and downsampling in time- and frequency domain   \\
%82 & multirate system analysis   \\
%83 & linearity of up- and downsampling systems   \\
%84 & filter bank   \\
%85 & interpolated FIR filter (IFIR), FIR window method design  \\
%\end{supertabular}
%\end{center}


\subsubsection{Matematiikka}
\label{sec:esimmatematiikka}

Lyhyet matemaattiset kaavat voi kirjoittaa tekstin
sisään $E_{\textrm{total}} = m_i c^2$, mutta kaavat, joita käytetään, 
kannattaa keskittää
%
\begin{equation}
\label{eq:kaava1}
x^2 + y^2 = 1 
\end{equation}
%
josta lyhyempi versio ilman kaavan numerointia
%
\[ x^2 + y^2 = 1 \]
%
tai jakaa useammalle riville
\begin{equation}
\label{eq:kaava2}
\begin{aligned}
x^2 + y^2 &= 1 \\
        x &= \sqrt{1-y^2}
\end{aligned}
\end{equation}
%

Kreikkalaiset kirjaimet löytyvät taulukosta~\ref{table:kreikka}.

\begin{table}
\caption{Kreikkalaiset kirjaimet}
\label{table:kreikka}
\begin{center}
\begin{tabular}{|llllllll|}
\hline
$\alpha$        &$\theta$       &o              &$\tau$         &%
$\beta$         &$\vartheta$    &$\pi$          &$\upsilon$     \\
$\gamma$        &$\gamma$       &$\varpi$       &$\phi$         &%
$\delta$        &$\kappa$       &$\rho$         &$\varphi$      \\
$\epsilon$      &$\lambda$      &$\varrho$      &$\chi$         &%
$\varepsilon$   &$\mu$          &$\sigma$       &$\psi$         \\
$\zeta$         &$\nu$          &$\varsigma$    &$\omega$       &%
$\eta$          &$\xi$          &               &               \\
\hline
$\Gamma$        &$\Lambda$      &$\Sigma$       &$\Psi$         &%
$\Delta$        &$\Xi$          &$\Upsilon$     &$\Omega$       \\
$\Theta$        &$\Pi$          &$\Phi$         & & & & &       \\
\hline
\end{tabular}
\end{center}
\end{table}

Matematiikkaan liittyviä ohjeistusta löytyy esim. \citet{lyhyt2e}.
Makroja sisältävästä tiedostosta \verb!makroja.tex! löytyy joitakin
esimerkkejä, kuten 
%
\[ \myInt{-\infty}{0}{e^{x}}{x} \]
%

\subsubsection{Algoritmit ja ohjelmalistaukset}

Työlle oleellisen tulostuslistauksen voi laittaa
\verb!verbatim!-ympäristöön.
%
\begin{verbatim}
Output written on main.pdf (23 pages, 268760 bytes).
Transcript written on main.log.
\end{verbatim}
%
\begin{sloppypar}
Algoritmien ja pseudokoodin esittämiseen tarvitaan esimerkiksi
\verb!algorithmic!- ja \verb!algorithm!-paketit.  Ohjelman esittelyn
voi tehdä vaikkapa \verb!program!-paketin avulla.  Ohjelmakoodia ei
tyypillisesti lisätä edes liitteeksi.  Jos näin kuitenkin tehdään,
käytä ylläolevia tai esimerkiksi \verb!listinginput!-komentoa. Näiden
käyttöön löytyy apua Internetistä.
% Katso esim. http://en.wikibooks.org/wiki/LaTeX/Algorithms_and_Pseudocode
\end{sloppypar}

\subsection{Viittaukset ja lähdeluettelo}
\label{sec:esimviitteet}

\subsubsection{Lähdetiedosto}
\label{sec:esimlahdetiedosto}

Lähdeluettelo kirjoitetaan joko käsin tai automaattisesti
kerätyistä lähteistä bib-päät\-tei\-seen tiedostoon. Katso 
esimerkkejä tiedostosta \verb!lahteet.bib!, jota kutsutaan
\verb!main.tex! tiedoston loppupuolella. 

Kirjaa tiedot mahdollisimman täydellisesti. Kiinnitä erityisesti
huomio nimiin ja kirjoita ne samanmuotoisesti
\verb!Sukunimi, Etunimi and Sukunimi2, Etunimi2 Etukirjain2.!.

Jos esimerkiksi kirjojen nimissä on muotoiluja, pakkaa ne
kaarisulkuihin \verb!{My {T}hesis}!. 
Pelkkä \verb!title={My Thesis}! muuttuu muuten
muotoon ``My thesis''.

Kurssilla opetetaan käyttämään RefWorksia lähteiden kokoamiseen.
Sieltä on mahdollista saada BibTeX-muotoinen luettelo lähteistä.
Valitse ylävalikosta ``Bibliography'' ja ensimmäisellä kerralla
valitse alasvetovalikosta ``Access output style manager'' ja
sieltä lisää BibTeX suosikkeihin. Määrää lähdetiedosto tekstityyppiseksi
(``Text'') ja luo tiedosto. RefWorksin tulostukseen tulee loppuun
ylimääräinen sulkumerkki \verb!}!, viimeisen tietorivin perässä
ylimääräinen pilkku ja alusta puuttuu itse viittausnimi, jonka
voit itse keksiä.

Google Scholar palauttaa myös BibTeX-muotoisia lähdeviitteitä,
kunhan olet sen asetuksissa valinnut BibTeXin mahdolliseksi.

\subsubsection{Tekstiviite}
\label{sec:esimtekstiviite}

Tässä pohjassa käytetään Helsingin yliopiston 
\verb!tktl!-tyyliä, joka pohjautuu \verb!alpha!-
ja \verb!natbib!-tyyliin~\cite{tktl}.
Tekstiviitteet saadaan mukaan joko tekijä-vuosi-tavalla (oletusarvo)
tai yksinkertaisella numeromerkinnällä, kuten \verb![1]!. 
Jälkimmäistä varten sinun pitää vaihtaa tekstiviitteen esitystapa
tiedon \verb!main.tex! lopussa rivillä \verb!bibpunct!. 

% kokeillaan vähän vapaampaa, kun natbibiin kohdalla rivivaihto
\begin{sloppypar}
Katso eri tapoja tekstiviitteiden muotoiluun sivulta
\url{http://merkel.zoneo.net/Latex/natbib.php}.
Viittauskomennot  \verb!\citet! ja \verb!\citep! ovat
hyviä tekijä-vuosi-tavassa \verb!natbib!iin liittyen
ja \verb!\cite! on perusviittauskomento.
\end{sloppypar}

Pari esimerkkiä:
% \citet
\citet[s. 21]{Teekkari2010} on havainnut asian jos toisenkin. 
% \citep
Tämä ja tuo uusi havainto on vahvistanut teoriaa~\citep[s. 22]{Teekkari2010}.
% \citep
Joku asia on selitetty tarkasti monissa lähteissä 
\citep[katso][s. 27]{Teekkari2010}.
% KATSO LISÄÄ: http://merkel.zoneo.net/Latex/natbib.php

\verb!tktl!-tyylin lähdetiedostoa \verb!tktl.dtx! ei ole
muokattu Aalto-yliopiston käyttöön, joten esimerkiksi 
lähdeviitteen merkintätyyppi \verb!@MasterThesis! tuottaa lähdeluetteloon
tekstin ``Pro gradu'' (kts. \verb!finnbst.tex!).
Tämä voidaan muuttaa antamalla kyseisen merkintätyypin kentälle
\verb!type! arvo ``Diplomityö''. 

\subsubsection{Lähdeluettelo}
\label{sec:esimlahdeluettelo}

Lähdeluettelon tulisi nyt ilmaantua dokumentin loppuun
automaattisesti. Jos sitä ei näy ja etenkin jos tekstiviitteiden
paikalla on kysymysmerkkejä, niin muista ajaa \verb!bibtex main!.

% --------------------------------------------------------------------

% Tätä käytetään vain testaustarkoituksiin, kun halutaan katsoa 
% miten sivu täyttyy pelkästä tekstistä tai miten fonttikoko
% vaikuttaa. 
%
% Teksti haettu eduskunta.fi-sivustolta tammikuussa 2011.

\section{Testi: pelkkää tekstiä}

Eduskunnan ympäristövaliokunta ehdottaa, että nykyinen
hajajätevesiasetus kumotaan ja ympäristöministeriö laatii uuden
asetuksen mahdollisimman pian. Lisäksi valiokunta ehdottaa, että
ympäristönsuojelulakiin sisällytetään uusi 3 a luku, johon tulee
yhteensä 5 pykälää. Ympäristövaliokunta antoi asiaa koskevan mietinnön
25. tammikuuta (YmVM 18/2010).

Valiokunta esittää lisäksi eduskunnan hyväksyttäväksi viisi
lausumaehdotusta, joilla vauhditetaan lain ja uuden asetuksen selkeää
ja tehokasta toimeenpanoa. Nyt esitettävät muutokset on valmisteltu
tiiviissä yhteistyössä ympäristöministeriön
kanssa. Ympäristövaliokunta on käsitellyt asiaa laajasti ja
perusteellisesti sekä ottanut ehdotuksissaan huomioon sen, mitä
perustuslakivaliokunta aiemmin asiassa edellytti.

Ehdotetuilla muutoksilla kohtuullistetaan hajajätevesien käsittelyn
vaatimustasoa siten, että vaatimukset asettuvat yleisesti sellaiselle
tasolle, että ne ovat kohtuullisella investoinnilla ja toimivalla
tekniikalla moitteettomasti täytettävissä. Esitetty kohtuullistaminen
ei kuitenkaan vaaranna ympäristönsuojelun tasoa ja on esimerkiksi
Itämeren suojelukomissio HELCOMin suositusten mukainen.

Nykyisen asetuksen lievempi vaatimustaso (orgaaninen aine 80\%,
kokonaisfosfori 70\%, kokonaistyppi 30\%) säädetään pääsääntöisesti
noudatettavaksi lähtökohdaksi, koska suuri osa kiinteistöistä
sijaitsee muualla kuin herkillä alueilla kuten ranta-alueella. Kunnat
voivat kuitenkin ympäristönsuojelumääräyksillä antaa tiukempia
määräyksiä ympäristön pilaantumisvaaran perusteella herkillä alueilla
kuten ranta-alueilla tai tärkeillä pohjavesialueilla.

Puhdistustasovaatimuksen perusteista säädetään laissa ja prosentit
edelleen asetuksella. Uudet säännökset ovat ensisijaisesti
jätevesijärjestelmän suunnittelun ja rakentamisen lähtökohta, eivät
valvontaperuste. Tosiasiallinen puhdistustulos voi siten vaihdella
esimerkiksi sääoloista tai kiinteistön käytön väliaikaisista
muutoksista johtuen, ja silti järjestelmä täyttää lain vaatimukset.

Valiokunta korostaa, että lain ja asetuksen mukaisten vaatimusten
toteuttaminen käytännössä edellyttää aina kiinteistökohtaista
arviointia. Kiinteistökohtainen neuvonta on siksi järjestettävä
valtakunnallisessa ohjauksessa. Valiokunta edellyttää riittävää
määrärahaa neuvonnan järjestämiseksi kunnissa.  Vaatimustason
muutoksella kohtuullistetaan tarvittavia investointeja, mutta
säilytetään samalla riittävä ympäristönsuojelun taso. Erityisen
tärkeää on ehkäistä lähiympäristön kuten kaivoveden pilaantumista ja
muita vastaavia hygieenisiä haittoja.

Muutoksella vapautetaan suoraan lain nojalla lain voimaan tullessa 68
vuotta täyttäneet vakituisesti asuttujen kiinteistön haltijat
asetuksen käsittelyvaatimuksista. Vapautus ei koske uudisrakentamista
eikä vapaa-ajan asuntoja. Käytännössä toimenpiteitä tarvitaan
vesivessalla varustetuilla vapaa-ajan asunnoilla.  Valtaosalla
vapaa-ajan asuntoja lainsäädäntö ei aiheuta mitään toimenpiteitä.

Muutoksella tarkennetaan, ketkä voivat hakea kunnalta viiden vuoden
mittaista vapautusta asetuksen vaatimusten noudattamisesta niin
sanotun sosiaalisen suoritusesteen (erityisen vaikeassa
elämäntilanteessa olevat kiinteistönomistajat kuten työttömät ja
pitkäaikaissairaat) perusteella.  Lainmuutos ja uusi asetus voivat
tulla voimaan 15.3.2011. Jo rakennetun kiinteistön olemassa olevan
jätevesijärjestelmän on täytettävä puhdistustehosta asetetut
vaatimukset vuoteen 2016 mennessä.

Suomen Euroopan neuvoston valtuuskunnan jäsenet Kimmo Sasi (kok.) ja
Krista Kiuru (sd.) vaativat parlamentaarisen yleiskokouksen
istunnossa, että Kosovossa rikosten uhreina kadonneiden ihmisten
kohtalot selvitetään.

Yleiskokous käsitteli tiistaiaamuna etukäteen paljon huomiota
herättänyttä sveitsiläisen kansanedustajan Dick Martyn raporttia
Kosovon vuoden 1999 sodan jälkimainingeissa tapahtuneista
ihmisoikeusloukkauksista. Martyn mukaan Kosovon vapautusarmeija
surmasi vangeiksi ottamiaan ihmisiä, poisti heiltä elimiä ja kauppasi
niitä pimeillä markkinoilla.

Keskustelussa Martyn väitteet myös kiistettiin. Kansanedustaja Krista
Kiuru korostikin täysistuntopuheessaan, että väitteet pitää tutkia
huolellisesti.  Totuuden selvittämiseksi tarvitaan puolueeton ja
läpinäkyvä tutkinta kansallisten ja kansainvälisten viranomaisten
yhteistyönä, hän totesi.  Kansanedustaja Kimmo Sasi (kok.) piti
huolestuttavana, että satojen ihmisten epäillään kadonneen sodan
jälkimainingeissa rikoksen uhreina.

Tarpeellisia selvityksiä ei voida tehdä ilman Albanian ja Kosovon
täydellistä yhteistyötä. Euroopan neuvoston pitääkin varmistaa, että
selvityksiä ei estetä tai vaikeuteta poliittisista syistä. Kosovon
hallitukselle onkin tärkeää, että asia tutkitaan. Vaikeidenkin
historian tapahtumien täydellinen läpikäynti auttaa rakentamaan
parempaa tulevaisuutta. Osoittamalla olevansa vahva oikeusvaltio,
vahvistaa Kosovo omaa asemaansa itsenäisenä valtiona, Sasi totesi.

Eduskunta korjaa pikavauhtia neljän kansanedustajan lakialoitteella
metsälakiin jääneen virheen, jonka takia pienet taimikot jäivät
tilapäisesti hirvivahinkokorvausten ulkopuolelle.  Metsälain muutos
tuli voimaan vuoden vaihteessa. Maa- ja metsätalousministeriössä lain
valmistelussa tapahtuneen teknisen virheen takia pienet, alle 1,3
metrin pituiset taimikot jäivät lakimuutoksen myötä hirvivahinkojen
korvauksen ulkopuolelle.  Lakitekstiin kiireessä jäänyt virhe
havaittiin vasta, kun laki oli jo hyväksytty ja vahvistettu, mitä
ministeriö pahoitteli. Virhe korjataan eduskunnan maa- ja
metsätalousvaliokunnan puheenjohtajan Jari Lepän (kesk.)
lakialoitteella, jonka on allekirjoittanut myös kolme muuta
valiokunnan kansanedustajaa.

Aloitteen tekijät ehdottavat, että kyseistä riistavahinkolain pykälää
korjattaisiin siten, että lakia sovellettaisiin taannehtivasti vuoden
alusta alkaen. Siten kukaan ei lopullisesti menetä oikeuttaan
korvaukseen.  Lakialoite (LA 121/2010) oli lähetekeskustelussa
tiistaina. Keskustelun päätteeksi asia lähetettiin maa- ja
metsätalousvaliokuntaan.

Euroopan unionin ympäristömerkki siirtyy samaan kotipesään
pohjoismaisen Joutsenmerkin kanssa. Eduskunta hyväksyi tiistaina
lakimuutoksen, jolla ympäristömerkinnän kansallisten tehtävien hoito
siirtyy Suomen Standardisoimisliitolta Motiva Services Oy:lle.
Suomessa on käytössä kaksi virallista
ympäristömerkintäjärjestelmää. Pohjoismainen ympäristömerkki eli
Joutsenmerkki on Pohjoismaiden ministerineuvoston vuonna 1989
perustama merkki. EU:n ympäristömerkki on Euroopan parlamentin ja
neuvoston antamaan asetukseen pohjautuva ympäristömerkki.

Merkit ovat vapaaehtoisia ja niillä kannustetaan valmistajia ja
palvelutarjoajia kehittämään ympäristön kannalta parempia
vaihtoehtoja. Joutsenmerkkejä on myönnetty yli 300. EU:n
ympäristömerkin luvanhaltijoita on noin 10 ja merkittyjä tuotteita
noin 50.  Joutsenmerkin hallinnointi siirtyi Motivalle vuoden
vaihteessa. Motiva Services Oy on valtion omistama asiantuntijayritys,
joka edistää energian ja materiaalien tehokasta ja kestävää käyttöä.

% --------------------------------------------------------------------

\section{Loppuluku}

Loppuluku päättää työn. Luvun nimi on tyypillisesti ``yhteenveto'' tai
``johtopäätöksiä''. Valitse se otsikko, joka tuntuu sopivammalta työsi
luonteeseen. Joka tapauksessa loppuluku sisältää niin työn yhteenvedon
kuin johtopäätöksiä työn tulosten perusteella. Pääajatus on antaa
lukijalle selvä kuva siitä, miten johdannossa asetettuihin
tavoitteisiin työssä vastattiin.

Käsittele loppupuvussa seuraavia asioita (jotakuinkin tässä järjestyksessä):
%
\begin{itemize}
  \item Muistutus työn tavoitteista (sidoksisuus johdantoon)
  \item Päätulokset kootaan yhteen, pohditaan niiden merkitystä
  \item Suositukset konkreettisiksi toimenpiteiksi (``Mitä sitten?'' 
Nyt kun käytössä on tämän työn myötä tullut tieto, 
mitä se nyt tarkoittaa tälle asialle/alalle.)
  \item Tulosten soveltuvuus, käyttöön liittyvät rajoitukset
  \item Jatkotutkimustarve 
(``Tulevaisuudessa olisi mielenkiintoista selvittää...'' tms.)
  \item Työn onnistumisen arviointi 
(Huom! Älä arvioi omaa kirjoitusprosessiasi vaan tekemääsi tutkimusta)
\end{itemize}

% --------------------------------------------------------------------

