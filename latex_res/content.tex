% --------------------------------------------------------------------

\section{Introduction}

In recent years, there have been rapid advances in the algorithms for combinatorial problems. 
This has been greatly sparked by the development in algebraic methods for solving the 
multilinear monomial detection problem, i.e., finding whether a multivariate polynomial contains a multilinear monomial. 
Namely, the technique of algebraic fingerprinting 
first introduced by Koutis in \cite{10.1007/978-3-540-70575-8_47} and further developed by Williams in \cite{WILLIAMS2009315} has proven 
to be very successful.\nl

With algebraic fingerprinting, the $k$-path problem (see problem def), 
that previously could be solved in \bigO{something} time by X in [ref], 
could be solved in \bigOstar{2^{3k/2}} time in \cite{10.1007/978-3-540-70575-8_47}. 
This result was quickly improved in \cite{WILLIAMS2009315}, where a \bigOstar{2^k} algorithm was given.\nl

Of course, this technique has been further developed, and in \cite{doi:10.1137/110839229} Björklund et al. showed an algorithm 
that solved the Hamiltonian problem (Hamiltonicity), i.e., finding whether a given graph contains a path that visits 
every vertex once, in \bigOstar{something} time. The previous fastest algorithm for Hamiltonicity by Y 
in [ref] ran in \bigO{something} time with the use of color coding[??]. [INSERT PREVIOUS MTEHOD] 
This was a significant improvement on a problem that had seen no progress in nearly fifty years.\nl

The technique of algebraic fingerprinting is present in multilinear monomial detection. Multilinear monomial detection 
is a fundamental problem, since many important combinatorial problems can be reduced into multilinear monomial detection 
via a problem specific algebraization. The goal of such algebraization is to form a multivariate polynomial 
that encodes the combinations, i.e. the solutions and non-solutions, into multivariate monomials where 
multilinear monomials correspond to solutions to the given problem.\nl

Before discussing multilinear monomial detection and algebraic fingerprints, the thesis covers algebraization, and 
shows how a combinatorial problem can be reduced into a multilinear monomial detection problem. 

[TODO: go through the structure of the thesis]

\subsection{Algebraization of combinatorial problems}

A combinatorial problem asks whether a given finite set of objects satisfies some given constraints. 
For example, the k-path problem asks for, given a finite set of vertices and edges, 
a path of k vertices. The solutions and non-solutions to combinatorial problems can be thought of as 
combinations of the given objects. The solution space for the k-path problem consists of combinations of k vertices and k-1 edges. 
A non-solution combination would contain duplicate vertices or edges that contain vertices outside the combination.\nl

Algebraization is reducing a given problem into an algebraic problem, i.e., a question regarding some algebraic property of some algebraic entity. 
In an algebraization of a combinatorial problem, the algebraic entity can be constructed from algebraic elements defined from the 
set of objects given as an input. The motivation behind the construction is some algebraic property that, 
when satisfied, gives a solution to the problem.\nl

[TODO: rewrite this paragraph, explain with generating polynomial and expasion into sum of monomials]
Multilinear monomial detection has proven to be a useful algebraization. First, we introduce multiple variables that 
correspond to elements from the set of objects given as input. Then, we construct a multivariate polynomial such that it 
encodes all solutions and non-solutions as multivariate monomials, with solutions encoded specifically as multilinear monomials. 
Thus, the task of finding a satisfying combination to the combinatorial problem has been reduced to 
finding a multilinear monomial from the multivariate polynomial. It follows, that a decision problem is answered by 
the existence of a multilinear monomial and a counting problem by the number of multilinear monomials.\nl

Appropriate definitions for the variables are problem specific. In the following section, this thesis shows a 
reduction into multilinear monomial detection, introduced in \cite{10.1145/2742544}, 
for the $k$-3D matching problem.

\subsection{Reducing $k$-3D matching into multilinear monomial detection}

The $k$-3D matching problem is defined as follows:

\begin{problem}
  \problemtitle{$k$-\textsc{3D matching}}
  \probleminput{Three disjoint sets $A$, $B$ and $C$, and a set of triples $T\subset A\times B\times C$.}
  \problemquestion{Is there a subset $M\subseteq T$, such that $\abs{M} = k$ and 
  $\forall m \in M$: None of the elements in $m$ appear in $M\backslash \{m\}$?}
\end{problem}

We begin by defining new variables corresponding to the elements in $A$, $B$ and $C$, 
labeled as $a_i$, $b_j$ and $c_k$, respectively, where $i\in [\abs{A}]$, $j\in [\abs{B}]$ and $k\in [\abs{C}]$. 

For every triple $t \in T$, we define a multilinear monomial $x$ that is a product of the elements in $t$. 
We introduce a set $X$ that satisfies the following:
\begin{center}$\forall x \in X$: $x = abc$ : $(a, b, c) \in T$.\end{center}

Next, we define multivariate polynomials $P_1$ and $P_k$ as follows:
\begin{center}$P_1 = \displaystyle \sum_{X}$ ,   $P_k = P_1^k$.\end{center}

Following this construction, we observe that $P_k$, when expanded into a sum of multivariate monomials, 
contains a multilinear term if and only if the original $k$-3D matching instance can be answered in the positive. 
Furthermore, every multilinear monomial in the expanded $P_k$ corresponds to a solution to the problem, and 
the solutions can be directly found from the variables in the multilinear monomial. Thus, 
a successful reduction into multilinear monomial detection has been given for the $k$-3D mapping.\nl

An example instance of $k$-3D matching with this exact algebraization can be found in [10.1145/2742544]. [show the example?]\nl


%A faster algorithm for an instance of multilinear monomial detection is likely to inspire faster algorithms for 
%other multilinear monomial detection problems [EXAMPLE HERE]. Of course, simply expanding the generating polynomial into 
%a sum of monomials is not an optimal solution.[Are there any other methods that can be applied generally?]\nl

%When the problem domain has n variables in an N-degree polynomial, the number of possible monomials is \(\binom{n+N}{n}\).
%This motivates the detection of multilinear monomials without fully expanding the polynomial into a sum of monomials, 
%which will be the topic of the next section.

\section{Preliminaries}

%[TODO: groups, rings, fields, characteristic of field, group algebra, 
%FPT \& O*, multivariate \& multilinear, generating \& sum of monomials form, 
%non-deterministic algorithm, polynomial ring]
It is necessary to recall basic algebraic concepts 
before further discussing multilinear monomial detections and algebraic fingerprinting. 
In this section, definitions for a group, ring and field are given, and some useful concepts 
regarding them. The second subsection goes through general notation and terminology used 
throughout the thesis.

\subsection{Groups, rings and fields} %polynomial rings, group algebra, char of field

A group $\textbf G$ is a tuple $(G, +)$, where $G$ is a set of elements, $+$ : $G \times G \longrightarrow G$ is a binary operation closed under 
the elements in $G$, $+$ is associative, every element $g\in G$ has an inverse $g^{-1}\in G$, and $G$ contains 
an identity element $e$ such that $g + e = g$, $g + g^{-1} = e$ and $e = e^{-1}$. Moreover, $\textbf G$ is called $Abelian$ if 
$+$ is also commutative.\nl

A ring $\textbf R$ is a tuple $(R, \cdot )$, where $R = (G, +)$ is an Abelian group, $\cdot$ : $G \times G \longrightarrow G$ 
is a binary operation closed under $G$. We call the binary operations $+$ and $\cdot$ as addition and multiplication, respectively. 
Note, that from here on we use $R$ as the set of elements defined for $\textbf R$. 
In general, a bold typeface \textbf X represents a group, ring or field and $X$ its set of elements. 
$R$ must contain a multiplicative identity $\textbf 1 \in R$ such that $\forall a \in R$: $a \cdot \textbf 1 = a$. 
We notate the additive identity $e$ required for the group $R$ as $\textbf 0$ from here on. 
Observe, that for any $\textbf R \neq \{\textbf 0\}$, $\textbf 1 \neq \textbf 0$.  
Left and right distributive laws hold for rings, i.e., 
\begin{center}
  $\forall a, b, c \in R$: $a \cdot (b + c) = (a \cdot b) + (a \cdot c) \land (b + c) \cdot a = (b \cdot a) + (c \cdot a)$.
\end{center}
$u \in R$ is called $unit$ if it holds that $\exists v \in R$: $u \cdot v = v \cdot u = \textbf 1$, 
i.e., it has a multiplicative inverse $v \in R$.\nl

A field $\textbf F = (F, +, \cdot)$ is defined with the following conditions:
\begin{itemize}
  \item $(F, +)$ is an Abelian group
  \item $(F\backslash \{\textbf 0\}, \cdot )$ is an Abelian group
  \item Left and right distributive laws hold for \textbf F
\end{itemize}

Equivalently, a ring is a field if every non-zero element is unit, $\textbf 1 \neq \textbf 0$, and multiplication is commutative. 
The $characteristic$ of a field \textbf F is defined as follows:
%\begin{center}
  \begin{equation}
    char(\textbf F) =
      \begin{cases}
        min\{n \in \N : n \cdot \textbf 1 = \textbf 0 \}\\
        0 & \text{if such $n$ does not exist}\\
      \end{cases}       
  \end{equation}
%\end{center}

Note, that a field \textbf F with characeristic 2 satisfies the following:
\begin{center}
  $\forall u \in F$: $u + u = u \cdot (\textbf 1 + \textbf 1) = u \cdot \textbf 0 = \textbf 0$
\end{center}

TODO: polynomial rings, group algebra

\subsection{Notation and other terminology} %FPT, O*, Theta, non-deterministic & deterministic algorithm

TODO: create a table or like, list terms: 
multilinearity, multivariety, sum of monomials form \& generating form (arithmetic circuit) of polynomial, 
degree of multivariate monomial, $\mathcal{O}$, $\Theta$, FPT, $\mathcal{O}$*, determinism \& non-determinism

\section{General multilinear monomial detection}

The detection of multilinear monomials is a fundamental problem, 
since many important problems can be reduced to it [TODO: quick examples (just refs?)]. 
Therefore, any progress in general multilinear monomial detection directly implies 
faster algorithms for all problems, that are reduced to and solved with general multilinear monomial detection. 
The general, parameterized multilinear monomial problem is defined as follows: 

\begin{problem}
  \problemtitle{$k$-\textsc{multilinear monomial detection}}
  \probleminput{A commutative arithmetic circuit $A$ over a set of variables $X$.}
  \problemquestion{Does the sum of monomials form of the polynomial $P(X)$ represented by $A$ 
  contain a multilinear monomial of degree $k$?}
\end{problem}


%However, a faster algorithm for some instance of multilinear monomial detection does not directly imply faster 
%algorithms for other combinatorial problems, since the algebraization into multilinear monomial detection is 
%problem specific. On the other hand, many multilinear monomial detection problems employ similar techniques [TODO: quick examples (just refs?)]. 
%As a result, finding new techniques and ideas for some specific multilinear monomial detection is valuable. 
%This thesis discusses clever utilizations of algebraic fingerprints on specific problems in a later section. 
%In this section, we focus on the general multilinear monomial detection, which\nl

\subsection{Non-algebraic methods}

Of course, fast evaluation of the expanded polynomial is also an important aspect, since it contributes 
directly to the general multilinear monomial detection problem. However, a naive expansion and evaluation is non-optimal.
When the problem domain has n variables in an N-degree polynomial, the number of possible monomials is \(\binom{n+N}{n}\) $= 2^{\Theta(N)}$.\nl

This motivates the detection of multilinear monomials without fully expanding the polynomial into a sum of monomials. 
On this section, the thesis gives a quick overview on non-algebraic methods for 
multilinear monomial detection that are faster than naive expansion and evaluation. These methods, however, are 
outperformed by the purely algebraic technique of algebraic fingerprinting which is covered later. 

[TODO: motivate quickly why we still go through them anyway] 

[dynamic programming is used in random color coding, 
random assignment is used in algebraic fingerprinting,
dynamic programming's set rule of any var squared = 0 can be achieved with matrices]

\subsubsection{Dynamic programming for smart expansion of the polynomial}

In multilinear monomial detection, only the multilinear terms are important. This implies that any non-linear term can be discarded 
as soon as they are formed, since the degree will not decrease during the expansion of the polynomial. Therefore, a dynamic 
programming algorithm can be designed for the expansion with an additional rule: any squared variable can be instantly discarded, or 
in algebraic terms, set to zero.\nl

With this additional rule, the following is deduced: if there are no solutions to the original problem, 
the polynomial will identically evaluate to zero. Note, however, that the polynomial 
evaluating to zero does not imply that no solutions exist. Indeed, we will come across a problem where 
multilinear monomials cancel each other, and thus evaluate to zero (see chapther x).

\subsubsection{Non-deterministic color coding for faster evaluation}

[TODO: go over random assignment with colors, 
which results a polynomial of smaller domain (less variables), thus 
making the evaluation faster]

[TODO: end with hints toward matrix assignment (a non-zero matrix squared can equal zero)]

\subsection{Non-deterministic color coding with matrices}

[TODO: go through idea of matrix assignment and specifications for a suitable algebra, 
then lead to fingerprinting]

\subsection{Algebraic fingerprinting to prevent unwanted cancellation}



[TODO: add subsections for problem specific implementations with different utilizations of fingerprints]

\subsection{Limits of general multilinear monomial detection}

[TODO: explain the limit in general multilinear detection with this algebra 
(impossible to find better algebra than what is used for the current fastest k-mld algorithm)]

[TODO: lead to problem specific implementations 
(we can use fingerprinting cleverly, when we understand the underlying problem well)]

\section{Problem specific implementations of algebraic fingerprints}

\subsection{Fingerprinting for cancellation of non-solutions}

[TODO: go over Björklund et al. for k-path, managed to insert fingerprints such that non-solutions cancel]

