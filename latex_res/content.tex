% --------------------------------------------------------------------

\section{Introduction}

In recent years, there have been rapid advances in the algorithms for combinatorial problems. 
This has been greatly sparked by the development in algebraic techniques for solving the 
multilinear monomial detection problem, i.e., finding whether a multivariate polynomial contains a multilinear monomial, 
first introduced by \citeauthor{Koutis05} in \cite{Koutis05}, where the set packing problem is
reduced to multilinear monomial detection.

Namely, the technique of algebraic fingeprinting, first introduced in
\cite{Koutis08} and further developed in \cite{Williams09}, 
has found great success for many combinatorial problems. 
For example, with algebraic fingerprinting, the $k$-path problem 
that previously could be solved in \bigOstar{4^{k}} time by \citeauthor{Chen07} in \cite{Chen07}, 
could be solved in \bigOstar{2^{3k/2}} time in \cite{Koutis08}. 
\amnote{What is the meaning of \bigOstar{}?}
This result was quickly improved in \cite{Williams09}, where an \bigOstar{2^k} algorithm was given.

\amnote*[inline,nomargin]{This technique was further developed by
\citeauthor{Björklund14} \cite{Björklund14}, who showed ...}{
Of course, this technique was further developed, and soon after in
\cite{Björklund14} Björklund et al. showed an algorithm 
}
\amnote{Note use of \texttt{\textbackslash citeauthor}}
that solved the Hamiltonian problem (Hamiltonicity), i.e., finding whether a given graph contains a simple path that visits 
every vertex, in \bigOstar{1.657^n} time. Soon enough, for $k$-path, an \bigOstar{1.66^k} algorithm was found \cite{Björklund17}. 
\amnote{Did Björklund really develop the technique further? or did he only show
how to apply it to another problem?}
The fastest algorithms for Hamiltonicity before this ran in
\amnote*{$n$ or $k$?}{ \bigOstar{2^n} }
and were known since 1962 \cite{HelKar62}, \cite{Bellman62}.
This was a significant improvement on a problem that had seen no progress in nearly fifty years.

\subsection{Research goals and thesis structure}
\amnote[inline,nomargin]{This section can be shortened to one paragraph}

The goals of this thesis are to find out how multilinear monomial detection is relevant in 
combinatorial problems, and how algebraic fingerprints can be utilized to design faster 
algorithms for problems that use multilinear monomial detection. Also, interesting ideas 
regarding algebraic fingerprints are explored for.

Multilinear monomial detection 
is a fundamental problem, since many important combinatorial problems can be reduced into it 
via a problem specific algebraization. Thus, faster algorithms and new ideas for multilinear monomial detection 
are important.

Multilinear monomial detection is essentially searching for solutions among non-solutions, 
both of which are encoded as monomials in a polynomial. 
The technique of algebraic fingerprinting is present in multilinear monomial detection. 
With algebraic fingerprints, unwanted cancelation of solution monomials due to the characteristic of a field can be prevented. 
Moreover, algebraic fingerprints can be used to cancel non-solution monomials by abusing the characteristic.

\amnote*{No need for this since you already have ToC}{%
In the next subsections, the thesis discusses algebraization and reduction into multilinear monomial detection. 
The section 2 covers preliminaries. In section 3, the thesis discusses general multilinear monomial detection. 
In section 4, some problem specific instances of multilinear monomial detection are given, and clever 
utilizations of algebraic fingerprints are shown. Section 5 concludes the
thesis.
}

\subsection{Algebraization of combinatorial problems}
\amnote{Algebrization}

A combinatorial problem asks whether a given finite set of objects satisfies some given constraints. 
For example, the $k$-path problem asks for, given a finite set of vertices and edges, 
a simple path of $k$ vertices. The solutions and non-solutions (solution space) to combinatorial problems can be thought of as 
combinations of the given objects.
\amnote*{Probably one can drop the vertices and think only about edges?}{%
The solution space for the $k$-path problem
consists of combinations of $k$ vertices and $k-1$ edges. 
}
A non-solution combination would contain duplicate vertices, or edges that
contain vertices outside the combination.

\amnote[inline,nomargin]{This sounds like a very simple problem... But I guess
the catch is that we want an algorithm with complexity depending only on $k$?}

Algebraization is reducing a given problem into an algebraic form,
\amnote*[inline,nomargin]{that is}{i.e.}, a
question regarding some algebraic property of some algebraic entity. 
In an algebraization of a combinatorial problem, the algebraic entity can be constructed from algebraic elements defined from the 
set of objects given as an input. The motivation behind the construction is some algebraic property that, 
when satisfied, gives a solution to the problem.

\amnote{Somewhere we need to reference the reader to Sect.~2 for non-familiar
terms like multivariate, algebras, etc.}
In \cite{Valiant92}, it was observed that multivariate polynomials in certain algebras have natural combinatorial interpretations. 
Utilizing this idea, \cite{Koutis05} managed to reduce a combinatorial problem
into an algebraic form, that is, multilinear monomial detection. 
First, we introduce multiple variables that 
correspond to elements from the set of objects given as input. 
Then, we construct an arithmetic circuit representing a multivariate polynomial, such that it 
encodes all solutions and non-solutions as multivariate monomials, 
with multilinear monomials corresponding to solutions. 
Thus, the task of finding a satisfying combination to the combinatorial problem has been reduced to 
finding a multilinear monomial from the multivariate polynomial.
It follows that a decision problem is answered by 
the existence of a multilinear monomial, and a counting problem by the number of multilinear monomials.

Appropriate definitions for the variables are problem-specific. In the following section, this thesis gives a 
reduction into multilinear monomial detection, shown in \cite{KouWil15}, 
for the $k$-3D matching problem. Another simple example can be found, for the set packing problem, in \cite{Koutis05}.

\amnote[inline,nomargin]{I would merge the two sections and use the $k$-3D
matching as an example to the explanation above}

\subsection{Reducing $k$-3D matching into multilinear monomial detection}

The $k$-3D matching problem is defined as follows:

\begin{problem}
  \problemtitle{$k$-\textsc{3D matching}}
  \probleminput{Three disjoint sets $A$, $B$ and $C$, and a set of triples $T\subset A\times B\times C$.}
  \problemquestion{Is there a subset $M\subseteq T$, such that $\abs{M} = k$ and 
  $\forall m \in M$: None of the elements in $m$ appear in
  \amnote*[inline,nomargin]{$M \setminus \{m\}$}{$M\backslash \{m\}$}}
\end{problem}
\amnote{Use \texttt{\textbackslash setminus} instead of \texttt{\textbackslash
backslash}}
\amnote{$T = A \times B \times C$ not allowed?}
\amnote{Either convert $\forall$ into text or what follows into maths}

We begin by defining new variables corresponding to the elements in $A$, $B$ and $C$, 
labeled as $a_i$, $b_j$ and $c_k$, respectively, where $i\in [\abs{A}]$, $j\in
[\abs{B}]$ and $k\in [\abs{C}]$. 
\amnote{Def. of $\abs{\cdot}$?}

For every triple $t \in T$, we define a multilinear monomial $x$ that is a
product of the elements in $t$. 
\amnote{Over what object are we doing multiplication?}
We introduce a set $X$ that satisfies the following:
%\begin{center}$\forall x \in X$: $x = abc$ : $(a, b, c) \in T$.\end{center}
\[
\forall x \in X: x = abc : (a, b, c) \in T.
\]
\amnote{Use \textbackslash[ ... \textbackslash] for display math}
\amnote[inline,nomargin]{Probably meant $\forall x \in X, x = abc ...$?}

Next, we define multivariate polynomials $P_1$ and $P_k$ as follows:
\begin{center}$P_1 = \displaystyle \sum_{X}$ ,   $P_k = P_1^k$.\end{center}
\amnote{Meaning of $\sum_X$?}

Following this construction, we observe that $P_k$, when expanded into a sum of multivariate monomials, 
contains a multilinear term if and only if the original $k$-3D matching instance can be answered in the positive. 
Furthermore, every multilinear monomial in the expanded $P_k$ corresponds to a solution to the problem, and 
the solutions can be directly found from the variables in the multilinear monomial. Thus, 
a successful reduction into multilinear monomial detection has been given for the $k$-3D mapping.

An example instance of $k$-3D matching with this exact algebraization can be found in \cite{KouWil15}. [show the example?]


%A faster algorithm for an instance of multilinear monomial detection is likely to inspire faster algorithms for 
%other multilinear monomial detection problems [EXAMPLE HERE]. Of course, simply expanding the generating polynomial into 
%a sum of monomials is not an optimal solution.[Are there any other methods that can be applied generally?]

%When the problem domain has n variables in an N-degree polynomial, the number of possible monomials is \(\binom{n+N}{n}\).
%This motivates the detection of multilinear monomials without fully expanding the polynomial into a sum of monomials, 
%which will be the topic of the next section.

\section{Preliminaries}

%[TODO: groups, rings, fields, characteristic of field, group algebra, 
%FPT \& O*, multivariate \& multilinear, generating \& sum of monomials form, 
%non-deterministic algorithm, polynomial ring]
It is necessary to recall basic algebraic concepts 
before further discussing multilinear monomial detections and algebraic fingerprinting. 
This section gives definitions for a group, ring and field, and some useful concepts 
regarding them. The second subsection goes through other notation and terminology used 
throughout the thesis.

\subsection{Groups, rings and fields} %polynomial rings, group algebra, char of field

A group \amnote*{There is also \texttt{\textbackslash mathbf}}{$\textbf G$}
is a tuple $(G, +)$, where $G$ is a set of elements,
\amnote{See TeX source}
%$+$ : $G \times G \longrightarrow G$
$+ \colon G \times G \to G$
is a binary operation closed under 
the elements in $G$, $+$ is associative, every element $g\in G$ has an inverse $g^{-1}\in G$, and $G$ contains 
an identity element $e$ such that $g + e = g$, $g + g^{-1} = e$ and $e = e^{-1}$. Moreover, $\textbf G$ is called $Abelian$ if 
$+$ is also commutative.
\amnote{I think $e = e^{-1}$ follows from the other requirements (but that is
not so important)}

A ring $\textbf R$ is a tuple
\amnote*[inline,nomargin]{$(R,+,\cdot)$}{$(R, \cdot )$},
where $R = (G, +)$ is an Abelian group, $\cdot$ : $G \times G \longrightarrow G$ 
is a binary operation closed under $G$. We call the binary operations $+$ and
$\cdot$
%as
addition and multiplication, respectively. 
Note, that from here on we use $R$ as the set of elements defined for $\textbf R$. 
In general, a bold typeface \textbf X represents a group, ring or field and $X$ its set of elements. 
$R$ must contain a multiplicative identity $\textbf 1 \in R$ such that $\forall a \in R$: $a \cdot \textbf 1 = a$. 
We notate the additive identity $e$ required for the group as $\textbf 0$ from here on. 
Observe, that for any $R \neq \{\textbf 0\}$, $\textbf 1 \neq \textbf 0$.  
Left and right distributive laws hold for rings, i.e., 
\begin{center}
  $\forall a, b, c \in R$: $a \cdot (b + c) = (a \cdot b) + (a \cdot c) \land (b + c) \cdot a = (b \cdot a) + (c \cdot a)$.
\end{center}
$u \in R$ is called $unit$ if it holds that $\exists v \in R$: $u \cdot v = v \cdot u = \textbf 1$, 
i.e., it has a multiplicative inverse $v \in R$.

A field $\textbf F = (F, +, \cdot)$ is defined with the following conditions:
\begin{itemize}
  \item $(F, +)$ is an Abelian group
  \item $(F\backslash \{\textbf 0\}, \cdot )$ is an Abelian group
  \item Left and right distributive laws hold for \textbf F
\end{itemize}

Equivalently, a ring is a field if every non-zero element is unit, $\textbf 1 \neq \textbf 0$, and multiplication is commutative. 
The $characteristic$ of a field \textbf F is defined as follows:
%\begin{center}
  \begin{equation}
    char(\textbf F) =
      \begin{cases}
        min\{n \in \N : n \cdot \textbf 1 = \textbf 0 \}\\
        0 & \text{if such $n$ does not exist}\\
      \end{cases}       
  \end{equation}
%\end{center}

Note, that a field \textbf F with
\amnote*{characteristic}{characeristic} 2 satisfies the following:
\begin{center}
  $\forall u \in F$: $u + u = u \cdot (\textbf 1 + \textbf 1) = u \cdot \textbf 0 = \textbf 0$
\end{center}

TODO: group algebra, polynomial ring, linear dependency

\subsection{Notation and other terminology} %FPT, O*, Theta, non-deterministic & deterministic algorithm

TODO: create a table or like, list terms: 
multilinearity, multivariety, sum of monomials form \& generating form (arithmetic circuit) of polynomial, 
degree of multivariate monomial, $\mathcal{O}$, $\Theta$, FPT, $\mathcal{O}$*, determinism \& non-determinism [or use 'Monte Carlo' :)], 
Schwartz-Zippel lemma

\amnote[inline,nomargin]{Standard $\mathcal{O}$ and $\Theta$ notation should be
known}

\section{General multilinear monomial detection}

The detection of multilinear monomials is a fundamental problem, 
since many important problems can be reduced to it [TODO: quick examples (just refs?)]. 
Therefore, any progress in general multilinear monomial detection directly implies 
faster algorithms
\amnote*{These problems should \enquote{live} in their own subsection, then you
can just call them \enquote{problems from Sect.~X.Y}}{%
for all problems, that are reduced to and solved with general
multilinear monomial detection. 
}
The general, parameterized multilinear monomial problem is defined as follows: 

\begin{problem}
  \problemtitle{$k$-\textsc{multilinear monomial detection}}
  \probleminput{A commutative arithmetic circuit $A$ over a set of variables $X$ representing a polynomial $P(X)$.}
  \problemquestion{Does the polynomial $P(X)$ extended as a sum of monomials 
  contain a multilinear monomial of degree $k$?}
\end{problem}

\amnote*{This gives an upper bound for solving the problem, but it is generally
not optimal}{
A naive expansion of $A$ into $P(X)$ and evaluation of $P(X)$ is not optimal: 
an $N$-degree polynomial will have $2^{\Theta(N)}$ possible monomials, and most 
problems, that can use this algebraization, have been solved with faster
algorithms (TODO: quick example?). 
}
This motivates the detection of multilinear monomials without fully expanding the polynomial into a sum of monomials.

In the next subsection,
\amnote*{I am having a hard time figuring out the overall structure}{%
this thesis quickly overviews dynamic programming and 
color coding for $k$-path in the context of 
$k$-multilinear monomial detection.
}
Although color coding is outperformed by the 
purely algebraic technique of fingerprinting, color coding is given as valuable background information, 
since some ideas carry on to the algebraic methods. Moreover, 
it appears that most parameterized problems that are solved with color coding can be reduced to 
multilinear monomial detection, and thus have faster algorithms
%\cite{KouWil15}, \cite{Fomin17}.
\amnote*{\texttt{\textbackslash cite} can take multiple parameters, see source}{
\cite{KouWil15,Fomin17}.
}

\amnote*{Use \texttt{\textbackslash label} and \texttt{\textbackslash cref}}{
In a later subsection, 
the thesis covers color coding with matrices as a purely algebraic method, which leads into 
the technique of algebraic fingerprinting. Then, a general framework for algebraic fingerprinting is discussed. 
In the last subsection under this section, 
the thesis discusses some limits and cons in general multilinear monomial
detection.
}

%However, a faster algorithm for some instance of multilinear monomial detection does not directly imply faster 
%algorithms for other combinatorial problems, since the algebraization into multilinear monomial detection is 
%problem specific. On the other hand, many multilinear monomial detection problems employ similar techniques [TODO: quick examples (just refs?)]. 
%As a result, finding new techniques and ideas for some specific multilinear monomial detection is valuable. 
%This thesis discusses clever utilizations of algebraic fingerprints on specific problems in a later section. 

\subsection{Non-algebraic methods}

Before discussing color coding, the thesis briefly overviews dynamic programming in the context of 
multilinear monomial detection, since dynamic programming is used in color coding. Furthermore, 
\amnote*[inline,nomargin]{we discuss}{the thesis introduces}
an algebraically interesting idea of setting squared variables to zero.

\subsubsection{Dynamic programming for smart expansion of the polynomial}

Multilinear monomial detection can be solved with dynamic programming. 
Dynamic programming, used for e.g. Hamiltonicity in \cite{HelKar62}, is a method where the problem 
is recursively broken down into smaller subproblems. \cite{Fomin17} gives a dynamic programming algorithm 
for solving an instance of $k$-multilinear monomial detection in the problem context of 
\amnote*{??}{representative families for product families}.

In multilinear monomial detection, only the multilinear terms are important. This implies that any non-linear term can be discarded 
as soon as they are formed in the arithmetic circuit, since the arithmetic circuit will never decrease the degree of a monomial. 
Therefore, a smart algorithm can be designed for the expansion that applies an additional rule: 
any squared variable can be instantly discarded.

With dynamic programming, this can be implemented by redesigning the arithmetic gates in the arithmetic circuit. 
In \amnote*[inline,nomargin]{practice}{practise},
\amnote*{I am not quite sure what is going on here}{%
the algorithm would employ
special addition and multiplication gates that discard squared variables. 
Thus, the evaluation of the (sum of monomials form) polynomial is reduced to
just evaluating the multilinear terms. 
For a multivariate polynomial with $n$ variables, there are $2^n$ multilinear terms, which implies 
that the expansion can be ran in \bigOstar{2^n} time \cite{KouWil15}.
}

The rule of discarding squared variables can be written in algebraic form: 
a squared variable $x^2 \in X$ is set to the additive identity (zero) of the field $\textbf F$, 
where $X \subset F$. 
With this additional rule for multilinear monomial detection, the following is deduced: 
if there are no solutions to the original problem, 
the polynomial will identically evaluate to zero.
\amnote[inline,nomargin]{So algebraically we are computing over
$\mathbf{F}[x]/(x^2)$?}

Note, however, that the polynomial evaluating to zero does not
\amnote[inline,nomargin]{necessarily}
imply that no
solutions exist. 
Indeed, we will come across a problem where multilinear monomials cancel each other, 
and thus evaluate to zero (see Section X.X).

\subsubsection{Non-deterministic color coding for faster evaluation}

\amnote{I am not sure if this is not a too big of a detour}

TODO: go over random assignment with colors, 
which results a polynomial of smaller domain (less variables), thus 
making the evaluation faster

Color coding was introduced in \cite{Alon95} as a randomized method for subgraph problems. Among others, 
a non-deterministic algorithm for the $k$-path problem is given.
First, the thesis explains the idea behind the algorithm in \cite{Alon95}. 
After this, the idea is translated into multilinear monomial detection for easier relevance.

\begin{problem}
  \problemtitle{\textsc{(directed or undirected)} $k$-\textsc{path (decision and search)}}
  \probleminput{A (directed or undirected) graph $G=(V,E)$.}
  \problemquestion{Does $G$ contain a simple path on $k$ vertices? If so, give such a path.}
\end{problem}

TODO: go quickly over the algorithm, translate it in terms of multilinear monomial detection, 
give focus on the random assignment in order to reduce domain size (adds non-determinism, though)

TODO: end with hints toward algebraic assignment (for example, a non-zero matrix squared can equal zero)

\subsection{Color coding with algebra}

TODO: go through idea of algebraic assignment and specifications for a suitable algebra, 
then lead to fingerprinting (fingerprints solve an issue of unwanted cancellation due to characteristic)

\subsubsection{Algebraic fingerprinting to prevent unwanted cancellation}

Algebraic fingerprints are introduced to solve the problem of unwanted cancellation. 
To prevent the multilinear monomials from cancelling each other, 
we augment the polynomial with new unique indeterminates, i.e. fingerprints, 
such that every multilinear monomial in the expanded polynomial is unique. 
The squared variables, and thus non-solution monomials, will still vanish, but 
the multilinear terms will prevail.

However, introducing new indeterminates into the arithmetic circuit raises the dimensions of matrices, which 
implies much slower matrix multiplications (matrix multiplication is exponential??). 
\citeauthor{Koutis08} uses random assignment from [\{0, 1\}] in order to find an odd k-mld problem, which 
can be solved by previous methods, since there is no cancellation (multilinear terms have odd coefficients).



\amnote{What is cancellation?}
TODO: explain how fingerprints prevent cancellation, 
give the general algebraic framework that appears in \cite{Williams09}, 
(also polynomial identity testing)

\cite{Koutis08} uses commutative group algebras of $\Z^{k}_{2}$ (to have squared variables equal zero) 
for ODD k-multilinear detection (multilinear monomials have odd coefficients). randomized assignment, no false positives
\amnote{ODD $=$ ordered decision diagram?}

Koutis reduces an algebraization into ODD k-multilinear detection by randomly assigning fingerprints from {0,1} in the hopes 
of getting odd number of solutions (k-multilinear terms), 
i.e., hoping that for a pair that cancels each other, one of them gets removed by 0-assignment so that cancelation is prevented

\cite{Williams09} reduces multilinear detection (after assinging variables values from some algebra) 
to polynomial identity testing for a polynomial P(A) that has fingerprints as variables. 
Fingerprints are then assigned values from a field that has more elements than P(A) has roots, which 
means that P(A) evaluates to non-zero with high probability (if there are solutions)



\subsubsection{Finding the solution}

Multilinear monomial detection has only been given as a detector for a solution, i.e., a decision algorithm. 
Talk about actually finding the solution.

\cite{Koutis08} gives an algorithm that solves the decision problem for $k$-path. 
$k$-path is found with \bigOstar{n+min(k^2, m)} applications of the algorithm.

\cite{Williams09} solves the decision problem for $k$-path. \cite{Williams09} also gives an algorithm 
that finds a path when it is known that a $k$-path exists.

\cite{Björklund17} solves the decision problem for $k$-path starting at some vertex $s$. 
A $k$-path can be found by just applying the algorithm for every vertex.

\subsection{Limits of general multilinear monomial detection}

TODO: give some cons wrt. color coding (difficult derandomization, 
are we able to add weights for optimization problems?) %(cost-constrained mld?)

TODO: explain the limit in general multilinear detection with this algebraic framework 
(impossible to find better algebra than what is used for the current fastest k-mld algorithm), \cite{KouWil09}

TODO: lead to problem specific implementations 
(we can use fingerprinting cleverly, when we understand the underlying problem well)

\section{Problem-specific applications of algebraic fingerprints}

TODO: add subsections for problem specific implementations with different utilizations of fingerprints

\subsection{Fingerprinting for cancellation of non-solutions}

TODO: go over Björklund et al. for k-path or Hamiltonicity, managed to design fingerprints such that non-solutions cancel

\section{Conclusion}

TODO: conclude