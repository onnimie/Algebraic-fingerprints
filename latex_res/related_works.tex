
The importance of multilinear monomial detection lies in the fact 
that many combinatorial problems, as we see in this section, 
can be reduced to instances of it. 
Thus, solving the general $k$-multilinear monomial detection problem 
efficiently has great interest. 
Currently, the technique of algebraic fingerprints underlies the fastest algorithms 
for all the problems mentioned here. 
Note that only time complexity is discussed here. 
%However, though omitted, space complexities have also been improved with algebraic fingerprinting.

Ideas behind representing combinatorial problems with multivariate polynomials have 
been considered before, e.g., by \citeauthor{Valiant92} \cite{Valiant92} and 
\citeauthor{Koutis05} \cite{Koutis05}. However, 
\citeauthor{Koutis08} was the first one to introduce and apply multilinear monomial 
detection for combinatorial problems, namely the $k$-path and $m$-set $k$-packing problems 
\cite{Koutis08}. For $k$-path, this improved the runtime from 
\bigOstar{4^k} \cite{Chen07} to \bigOstar{2^{3k/2}}, and for $m$-set $k$-packing, 
the runtime was improved from \bigOstar{5.44^{mk}} \cite{Koutis05} to \bigOstar{2^{mk}}.

Soon after, \citeauthor{Williams09} developed this algebraic technique 
for an \bigOstar{2^k} $k$-path \cite{Williams09}, 
from which \citeauthor{KouWil09} \cite{KouWil09, KouWil15} 
gathered and proposed a general 
\bigOstar{2^k} algebraic framework, 
\emph{algebraic fingerprinting}, for $k$-multilinear monomial detection 
as a general technique for 
parameterized combinatorial problems. In their work \cite{KouWil09}, 
\citeauthor{KouWil09} gave faster \bigOstar{2^k} algorithms using this technique for the 
$k$-tree, $k$-leaf spanning tree, and $t$-dominating set problems. %, and an 
%\bigOstar{2^{(m-1)k}} algorithm for the $m$-dimensional $k$-matching problem. 
The runtimes of the previously fastest algorithms for these problems, respectively, were   
\bigOstar{(2e)^k} \cite{Alon95}, \bigOstar{4^k} \cite{Kneis11}, \bigOstar{(4+\varepsilon)^t} \cite{Kneis07}. 

Due to the development in algebraic fingerprinting, 
\citeauthor{Björklund14} \cite{Björklund14} found an \bigOstar{1.657^n} algorithm for 
Hamiltonicity, improving the runtime from \bigOstar{2^n} \cite{Bellman62, HelKar62}, 
by clever algebrization and utilizations of algebraic fingerprints. Using similar 
ideas, \citeauthor{Björklund17} \cite{Björklund17} gave faster algorithms for 
$k$-path, $m$-set $k$-packing, and its special case of 3-set $k$-packing. 
The runtimes for these were \bigOstar{1.657^k}, \bigOstar{2^{(m-2)k}} and \bigOstar{1.493^{3k}}, respectively.

$k$-multilinear monomial detection itself was widely studied by 
\citeauthor{Chen07} \cite{Chen07, Chen11a}, \citeauthor{Chen13a} \cite{Chen13a, Chen11b} 
and \citeauthor{Chen13b} \cite{Chen13b}. TODO: what did they research (deterministic stuff, primality of the field) 
(In this thesis though, we focus on the randomized technique of algebraic fingerprinting.)

TODO: deterministic multilinear detection

TODO: constrained multilinear detection (weights? functional motifs for biological networks)

TODO: parallelization (MIDAS?)

Stuff: Fast graph scan statistics, temporal graphs, 